\documentclass[]{article}
\hyphenation{co-rres-pon-dien-tes te-ner E-ben-sper-ger de-pre-da-do-res dis-po-ni-ble be-ne-fi-cio-sa in-di-vi-dual so-cia-li-dad mos-tra-ron fuen-tes a-cep-ta-ble ta-ma-ño o-pues-ta mo-de-lo es-tu-dian-tes e-jer-ci-cios co-rres-pon-dien-te mo-di-fi-ca-dos mo-di-fi-car-lo ma-ni-pu-lar}
\usepackage{amssymb,amsmath}
\usepackage{ifxetex,ifluatex}
\ifxetex
  \usepackage{fontspec,xltxtra,xunicode}
  \defaultfontfeatures{Mapping=tex-text,Scale=MatchLowercase}
\else
  \ifluatex
    \usepackage{fontspec}
    \defaultfontfeatures{Mapping=tex-text,Scale=MatchLowercase}
  \else
    \usepackage[utf8]{inputenc}
  \fi
\fi
\usepackage{color}
\usepackage{fancyvrb}
\DefineShortVerb[commandchars=\\\{\}]{\|}
\DefineVerbatimEnvironment{Highlighting}{Verbatim}{commandchars=\\\{\}}
% Add ',fontsize=\small' for more characters per line
\newenvironment{Shaded}{}{}
\newcommand{\KeywordTok}[1]{\textcolor[rgb]{0.00,0.44,0.13}{\textbf{{#1}}}}
\newcommand{\DataTypeTok}[1]{\textcolor[rgb]{0.56,0.13,0.00}{{#1}}}
\newcommand{\DecValTok}[1]{\textcolor[rgb]{0.25,0.63,0.44}{{#1}}}
\newcommand{\BaseNTok}[1]{\textcolor[rgb]{0.25,0.63,0.44}{{#1}}}
\newcommand{\FloatTok}[1]{\textcolor[rgb]{0.25,0.63,0.44}{{#1}}}
\newcommand{\CharTok}[1]{\textcolor[rgb]{0.25,0.44,0.63}{{#1}}}
\newcommand{\StringTok}[1]{\textcolor[rgb]{0.25,0.44,0.63}{{#1}}}
\newcommand{\CommentTok}[1]{\textcolor[rgb]{0.38,0.63,0.69}{\textit{{#1}}}}
\newcommand{\OtherTok}[1]{\textcolor[rgb]{0.00,0.44,0.13}{{#1}}}
\newcommand{\AlertTok}[1]{\textcolor[rgb]{1.00,0.00,0.00}{\textbf{{#1}}}}
\newcommand{\FunctionTok}[1]{\textcolor[rgb]{0.02,0.16,0.49}{{#1}}}
\newcommand{\RegionMarkerTok}[1]{{#1}}
\newcommand{\ErrorTok}[1]{\textcolor[rgb]{1.00,0.00,0.00}{\textbf{{#1}}}}
\newcommand{\NormalTok}[1]{{#1}}
% Redefine labelwidth for lists; otherwise, the enumerate package will cause
% markers to extend beyond the left margin.
\makeatletter\AtBeginDocument{%
  \renewcommand{\@listi}
    {\setlength{\labelwidth}{4em}}
}\makeatother
\usepackage{enumerate}
\usepackage{graphicx}
% We will generate all images so they have a width \maxwidth. This means
% that they will get their normal width if they fit onto the page, but
% are scaled down if they would overflow the margins.
\makeatletter
\def\maxwidth{\ifdim\Gin@nat@width>\linewidth\linewidth
\else\Gin@nat@width\fi}
\makeatother
\let\Oldincludegraphics\includegraphics
\renewcommand{\includegraphics}[1]{\Oldincludegraphics[width=\maxwidth]{#1}}
\ifxetex
  \usepackage[setpagesize=false, % page size defined by xetex
              unicode=false, % unicode breaks when used with xetex
              xetex,
              colorlinks=true,
              linkcolor=blue]{hyperref}
\else
  \usepackage[unicode=true,
              colorlinks=true,
              linkcolor=blue]{hyperref}
\fi
\hypersetup{breaklinks=true, pdfborder={0 0 0}}
\setlength{\parindent}{0pt}
\setlength{\parskip}{6pt plus 2pt minus 1pt}
\setlength{\emergencystretch}{3em}  % prevent overfull lines
\setcounter{secnumdepth}{0}


\begin{document}

\section{Ejercicios de programación III: Trabajo con datos}

\subsubsection{{[}IMSER 2012{]}}

\begin{center}\rule{3in}{0.4pt}\end{center}

\subsection{Archivos incluidos:}

El \href{http://goo.gl/eFUKY}{archivo} con los ejercicios del práctico
debe bajarse y descomprimirse en disco duro, creando la carpeta
\textbf{\texttt{rep-3}} (nota: no debe dentro de ningún disco, partición
o carpeta protegida a la escritura, como puede ser un disco duro externo
de backup). Usted deberá abrir el RStudio y seleccionar dicha carpeta
como su directorio de trabajo con \texttt{setwd} o en RStudio la
combinación \textbf{Ctrl + Shift + K}. En esta carpeta se encuentran
algunos archivos que usted deberá modificar:

\begin{itemize}
\item
  \textbf{\texttt{importar.R}}
\item
  \textbf{\texttt{parche.R}}
\item
  \textbf{\texttt{filtrado.R}}
\item
  \textbf{\texttt{est.R}}
\item
  \textbf{\texttt{transformar.R}}
\item
  \textbf{\texttt{nuevo-factor.R}}
\item
  \textbf{\texttt{exportar.R}}
\end{itemize}
Adicionalmente los siguientes archivos son necesarios, pero \textbf{no
deben ser modificados} para que el método de calificación automático
funcione correctamente.

\begin{itemize}
\item
  \texttt{evaluar.R}
\item
  \texttt{notas.csv}
\item
  \texttt{datos}
\item
  \texttt{INSTRUCCIONES.pdf}
\item
  \texttt{est.RData}
\end{itemize}
\subsection{Mecanismo de corrección:}

Nota: más recomendaciones \textbf{importantes} se hacen en el documento
\href{http://goo.gl/P5Wnq}{Dinámica de los repartidos}.

Lo primero que debe hacer es cargar el archivo evaluar.R con la función
\texttt{source} y la codificación de caracteres ``UTF-8'' (lo cual
afecta a la función \texttt{evaluar} en particular), de la siguiente
manera:

\begin{Shaded}
\begin{Highlighting}[]
\KeywordTok{options}\NormalTok{(}\DataTypeTok{encoding =} \StringTok{"utf-8"}\NormalTok{)}
\KeywordTok{source}\NormalTok{(}\StringTok{"evaluar.R"}\NormalTok{)}
\end{Highlighting}
\end{Shaded}
Si usted ha ejecutado todos los pasos anteriores correctamente, la
siguiente frase debería verse en la consola:

\begin{verbatim}
Archivo de codigo fuente cargado correctamente
\end{verbatim}
En caso de que ocurra un error o se vea otro mensaje en la consola,
verifique que los archivos se descomprimieron correctamente y que usted
está trabajando en la carpeta correspondiente con el comando
\texttt{getwd()}.

Usted trabajará modificando los contenidos de dichos archivos con
RStudio (u otro programa de su preferencia) según las consignas que se
describen a continuación. Luego de terminar cada ejercicio y
\textbf{guardando el archivo} correspondiente en el disco duro, usted
podrá verificar rápidamente si su respuesta es correcta ejecutando el
comando:

\begin{Shaded}
\begin{Highlighting}[]
\KeywordTok{evaluar}\NormalTok{()}
\end{Highlighting}
\end{Shaded}
y además podrá en todo momento verificar su puntaje con la función
\texttt{verNotas()}. Tenga siempre en cuenta que, a \textbf{menos que
sea indicado} por la letra del ejercicio, las soluciones deben ser
genéricas y por lo tanto deben servir aún si se modifican los datos
originales (i.e.: no use valores fijos si no comandos). Usualmente se
utilizan valores generados de forma aleatoria para las correcciones
automáticas. Los objetos que son evaluados en la corrección automática
estarán indicados con un asterísco en las instrucciones de cada script.
Nótese además que en los archivos \textbf{se indica claramente en dónde
se inicia y dónde finaliza su código} y que debe respetar esta
organización para que la corrección de los ejercicios funcione bien.

\subsubsection{Al finalizar}

Una vez terminados y guardados los archivos de los ejercicios del
repartido, usted deberá ejecutar \texttt{evaluar()} y seleccionar la
última opción (``Todos'') y luego subir el archivo ''datos'' (sin
extensión), incluido en la carpeta ''rep-1'', a la
\href{http://eva.universidad.edu.uy/mod/assign/view.php?id=122712}{sección
de entregas} de la portada del curso en la plataforma EVA. Este archivo
se podrá reemplazar con uno más nuevo, en caso de que desee corregir
algún error; en caso de querer que el archivo sea corregido antes de la
fecha de entrega, puede cambiarle el nombre a ``datos-finalizado'', pero
en ese caso la nota no se cambiará de ahí en adelante.

\subsubsection{Código de Honor}

Si bien animamos a que trabaje en equipos y que haya un intercambio
fluido en los foros del curso, es fundamental que las respuestas a los
cuestionarios y ejercicios de programación sean fruto del trabajo
individual. En particular, consideramos necesario que no utilice el
código creado por sus compañeros, si no que debe programar sus propias
instrucciones, ya que de lo contrario supone un sabotaje a su propio
proceso de aprendizaje. Esto implica también evitar, en la medida de lo
posible, exponer el código propio a sus colegas. Como profesores estamos
comprometidos a dar nuestro mayor esfuerzo para dar las herramientas y
explicaciones adecuadas a fin de que pueda encontrar su propio camino
para resolver los ejercicios.

En casos de planteos de dudas a través del foro, en los que considere
que es imposible expresar un problema sin exponer su própio código,
entonces es aceptable hacerlo. De todas formas en estos casos es
preferible que envíe su código por correo electrónico directamente a un
profesor, explicando la problemática.

\begin{center}\rule{3in}{0.4pt}\end{center}

\subsection{1. Datos de EEUU}

En la carpeta del repartido `rep-3' se encuentra una planilla de cálculo
en formato xls, llamada ``usa.xls''. Esta planilla tiene una serie de
variables medidas para los cincuenta estados de EE.UU., durante los años
setenta. Para mayor información de estos datos, ejecute el comando
\texttt{?state}.

A partir de esta base de datos vamos a trabajar a lo largo de todo el
repartido, ejercitando las habilidades y conocimientos necesarios para
modificar y manipular data.frames. Esta es la principal clase de objeto
para trabajar con datos que tiene R y por lo tanto nos centramos
principalmente en esta.

\subsubsection{1.a Importar los datos}

Script: ``importar.R''

Lo primero que se debe hacer es importar los datos a R. Para ello usted
deberá exportar la planilla de cálculo desde excell u otro programa
capaz de trabajar con ella. Dicha exportación deberá hacerse en un
archivo de texto plano (cuyas extensiones suelen ser .txt o .csv). En la
evaluación automática no se corregirá dicho archivo, pero es necesario
que se encuentre en la carpeta del repartido para que pueda ejecutarse
el script ``importar.R''.

En dicho archivo usted deberá crear el código necesario para importar
los datos a R usando alguna de las variantes de \texttt{read.table}. La
única condición importante es que la primer columna (los nombres de los
estados de EE.UU.) debe importarse como los nombres de filas de la
data.frame resultante en R. Dicha data.frame deberá llamarse
\texttt{usa} (ver instrucciones en el arcihvo ``importar.R'').

Pero además de importar los datos a su área de trabajo, el script debe
también cambiarle los nombres de las variables al español.
Específicamente, los nombres de las variables de la data.frame deben ser
(en el mismo orden y respetando mayúsculas y minúsculas):

\begin{verbatim}
Abrev, Poblacion, Ingresos, Analf, Esp.Vida, Homicidio, Sec.Grad, 
Heladas, Area, Division
\end{verbatim}
Por último, en la variable \texttt{Division} también queremos cambiar
los nombres de los 9 niveles (se trata de un factor) a una versión en
español. Específicamente (en el mismo orden y respetando mayúsculas y
minúsculas):

\begin{verbatim}
Noreste Central, Sudeste Central, Atlantico Central, Montania,
Nueva Inglaterra, Pacifico, Atlantico Sur, Noroeste Central,
Sudoeste Central
\end{verbatim}
Nótese que no se usan tildes ni eñes en los nombres para evitar
problemas de codificación de caracteres y que se deben respetar
mayúsculas y minúsculas.

(Pista: considere usar la función \texttt{levels} para esta tarea.)

\subsubsection{1.b Corregir datos de analfabetismo}

Script: ``parche.R''

Este ejercicio parte de la base que usted logró importar con éxito los
datos de ``usa.xls'' como se indica en el ejercicio anterior. Si usted
ejecuta ahora el comando

\begin{Shaded}
\begin{Highlighting}[]
\KeywordTok{summary}\NormalTok{(usa)}
\end{Highlighting}
\end{Shaded}
podrá notar que para las columnas \texttt{Ingresos} y \texttt{Analf}
figura un conteo de la cantidad de NA's. Esto quiere decir que hay datos
faltantes (``Not Available''; ver en
\href{http://eva.universidad.edu.uy/mod/glossary/view.php?id=116962}{el
glosario} para mayor información). Esto suele ser un problema para
trabajar con datos en general y por lo tanto es importante encontrar la
forma sortear este tipo de obstáculos.

Afortunadamente en este caso tenemos una tabla de datos auxiliar que nos
permite completar lo que nos falta para la columna \texttt{Analf} (tasa
de analfabetismo). Esta tabla auxiliar está en el archivo
``usa-extra.csv''. Para completar el ejercicio, usted deberá completar
todos los pasos necesarios para

\begin{enumerate}[1.]
\item
  importar estos datos,
\item
  seleccionar los valores de analfabetismo de los estados correctos y
\item
  sustituir los \texttt{NA} de la data.frame \texttt{usa2}, columna
  \texttt{Analf}, por estos datos seleccionados.
\end{enumerate}
Nótese que la data.frame \texttt{usa} debe permanecer incambiada y se
debe crear el objeto \texttt{usa2} para hacer estas modificaciones.
Nótese también que ``usa-extra.csv'' es una tabla muy distinta a la
planilla original, incluyendo sólo 2 columnas y menor cantidad de filas,
por lo que la única manera de determinar la ubicación de los valores
correctos es a través del uso de los nombres de los estados como
referencia. En este sentido es bueno recordar que el operador lógico
\texttt{\%in\%} puede ser de mucha utilidad; también es importante
recordar que este no es un operador conmutativo; es decir no es lo mismo

\begin{Shaded}
\begin{Highlighting}[]
\NormalTok{x %in% y}
\end{Highlighting}
\end{Shaded}
que

\begin{Shaded}
\begin{Highlighting}[]
\NormalTok{y %in% x.}
\end{Highlighting}
\end{Shaded}
El siguiente es un ejemplo que puede servir de guía:

\begin{Shaded}
\begin{Highlighting}[]
\NormalTok{a <- }\KeywordTok{c}\NormalTok{(}\StringTok{"ta"}\NormalTok{, }\StringTok{"te"}\NormalTok{, }\StringTok{"ti"}\NormalTok{)}
\NormalTok{b <- }\KeywordTok{c}\NormalTok{(}\StringTok{"ta"}\NormalTok{, }\StringTok{"ti"}\NormalTok{)}
\NormalTok{a %in% b}
\end{Highlighting}
\end{Shaded}
\begin{verbatim}
## [1]  TRUE FALSE  TRUE
\end{verbatim}
(Con esto se pueden obtener las posiciones en las que \texttt{a}
contiene los mismos elementos que \texttt{b}.)

\subsubsection{1.c Eliminar filas sin datos de ingresos}

Script: ``filtrado.R''

Así como para el analfabetismo tuvimos una forma de llenar un vacío de
datos, para el caso de la columna ``Ingresos'' no tenemos la misma
suerte. Por lo tanto, considerando que lo mejor es dejar de lado los
casos en que hay ausencia de datos, en esta parte del ejercicio vamos a
eliminar las filas correspondientes de ``usa2''.

Para esto usted deberá escribir el código necesario en el archivo
``filtrado.R''. Este código debe asumir la existencia de una data.frame
llamada \texttt{usa2}, y servirá para obtener finalmente una data.frame
\texttt{usa3} a través de la eliminación de las observaciones de
\texttt{usa2}, columna \texttt{Ingresos}, en las que ocurren valores
\texttt{NA}'s.

El siguiente es un mini ejemplo que puede servir como referencia. Aquí
le cambiamos a \texttt{NA} algunos valores a la una data.frame
\texttt{datos} y luego eliminamos las filas correspondientes.

\begin{Shaded}
\begin{Highlighting}[]
\NormalTok{datos <- }\KeywordTok{head}\NormalTok{(cars)}
\NormalTok{datos[}\KeywordTok{c}\NormalTok{(}\DecValTok{3}\NormalTok{, }\DecValTok{5}\NormalTok{), }\DecValTok{2}\NormalTok{] <- }\OtherTok{NA}

\CommentTok{# La data.frame antes:}
\NormalTok{datos}
\end{Highlighting}
\end{Shaded}
\begin{verbatim}
##   speed dist
## 1     4    2
## 2     4   10
## 3     7   NA
## 4     7   22
## 5     8   NA
## 6     9   10
\end{verbatim}
\begin{Shaded}
\begin{Highlighting}[]
\CommentTok{# La data.frame después:}
\NormalTok{datos.filtrado}
\end{Highlighting}
\end{Shaded}
\begin{verbatim}
##   speed dist
## 1     4    2
## 2     4   10
## 4     7   22
## 6     9   10
\end{verbatim}
\paragraph{Sugerencia:}

Utilizar la función \texttt{subset} para esta tarea.

\subsubsection{1.d Extra: función para estandarizar valores de un
vector}

(\emph{Este ejercicio es opcional, aunque puede sumar puntos en su
calificación final del repartido})

Script: ``est.R''

Muchas veces es útil al analizar datos transformar variables usando
distintas fórmulas. Una de ellas es la estandarización de datos,
utilizando la fórmula:

\[
  Z_i = \frac{X_i - \mu}{\sigma}
\]

En donde $X_i$ representa el iésimo de los datos originales, $\mu$ es el
valor promedio de todos los $X_i$, $\sigma$ es el desvío estándar de los
$X_i$ y los $Z_i$ son los valores estandarizados.

En este ejercicio usted deberá crear una función llamada \texttt{est}
(puede tomar como ejemplo las realizadas en el primer repartido u otras
mostradas en las lecciones) que tome como entrada \textbf{un sólo
argumento}: un vector numérico cualquiera y devuelva otro vector
numérico con los valores estandarizados del original. Para esto deberá
escribir el código necesario en el archivo ``est.R''.

Aconsejamos utilizar las funciones \texttt{mean} y \texttt{sd} para
obtener $\mu$ y $\sigma$ respectivamente. Además es deseable que las
normalizaciones de datos no se vean afectadas por la ocurrencia de
\texttt{NA}'s. Por lo tanto, es necesario utilizar el argumento
\texttt{na.rm} de dichas funciones para que \texttt{est} maneje
correctamente los \texttt{NA}'s. En caso de que la haya construido bien,
debería obtener resultados como el siguiente:

\begin{Shaded}
\begin{Highlighting}[]
\NormalTok{x <- }\KeywordTok{c}\NormalTok{(}\FloatTok{4.5}\NormalTok{, }\FloatTok{12.3}\NormalTok{, }\FloatTok{5.8}\NormalTok{, }\FloatTok{9.4}\NormalTok{, }\FloatTok{7.3}\NormalTok{, }\OtherTok{NA}\NormalTok{)}
\KeywordTok{est}\NormalTok{(x)}
\end{Highlighting}
\end{Shaded}
\begin{verbatim}
## [1] -1.0911  1.4418 -0.6690  0.5001 -0.1819      NA
\end{verbatim}
Nótese que si la función \texttt{est} no maneja correctamente los
\texttt{NA}'s, entonces el resultado sería igual a
\texttt{rep(NA, length(x))}.

\subsubsection{1.e Estandarizar los datos}

Script: ``transformar.R''

La estandarización o normalización es una tranformación común en el
análisis de datos. En general para cualquier tipo de transformación, si
se trata de un trabajo con matrices o data.frames, es común en R el uso
de las funciónes del tipo \texttt{apply}, ya que permiten modificar
varias columnas en un sólo comando y además pueden ser más eficientes
(en particular \texttt{lapply} o \texttt{sapply}) cuando se trabaja con
grandes cantidades de datos.

En este ejercicio el objetivo es usar la función \texttt{est} creada en
el ejercicio anterior, en conjunción con \texttt{apply}, para
transformar las columnas de clase ``numeric'' de nuestra data.frame
\texttt{usa3}. Específicamente, debe usar \texttt{apply} para
transformar el objeto \texttt{datosNumericos}, el cual están en el
script ``trasnformar.R''. En caso de \textbf{no haber hecho} el
ejercicio 1.d, puede cargar una función \texttt{est} hecha de antemano
con el comando:

\begin{Shaded}
\begin{Highlighting}[]
\KeywordTok{load}\NormalTok{(}\StringTok{"est.RData"}\NormalTok{)}
\end{Highlighting}
\end{Shaded}
En el script ``transformar.R'' se indica específicamente en qué línea
debe utilizarse \texttt{apply} para que la evaluación del ejercicio
funcione correctamente (i.e.: la línea en que se crea el objeto
\texttt{datosTrans}). Si usted ha creado correctamente el objeto
\texttt{datosTrans}, entonces este será de clase ``matrix'' y los
promedios de las columnas \texttt{Poblacion} y \texttt{Area} serán
respectivamente:

\begin{Shaded}
\begin{Highlighting}[]
\KeywordTok{colMeans}\NormalTok{(datosTrans[, }\KeywordTok{c}\NormalTok{(}\StringTok{'Poblacion'}\NormalTok{, }\StringTok{'Area'}\NormalTok{)])}
    \NormalTok{Poblacion          Area }
\NormalTok{-}\FloatTok{6.745250e-17} \NormalTok{-}\FloatTok{2.678736e-17}
\end{Highlighting}
\end{Shaded}
Finalmente, tome en cuenta también que el objeto final que usted debe
crear, llamado \texttt{usaNorm}, debe ser de clase `data.frame' y debe
tener las mismas columnas de clase ``factor'' del objeto inicial
\texttt{usa3}. También deben coincidir los nombres de filas y columnas.
Para esto recomendamos primero coercionar \texttt{usaNorm} en una
data.frame con la función correspondiente y luego unir el objeto
resultante con las columnas ``factor'' de \texttt{usa3} (siempre
manteniendo el orden de columnas).

\subsubsection{1.f Extra: un nuevo factor}

(\emph{Este ejercicio es opcional, aunque puede sumar puntos en su
calificación final del repartido})

Script: ``nuevo-factor.R''

En este ejercicio se propone crear una nueva columna de clase ``factor''
en la data.frame \texttt{usa3}, utilizando la función \texttt{cut}.
Dicho factor deberá llamarse \texttt{Ing.Cat} (como se ilustra en el
script), tener 4 niveles y ser construido en base a la columna
\texttt{Ingresos} de \texttt{usa3}. Esta variable representará entonces
las 4 categorías de ingreso (promedio, por estado) de EE.UU. Si el
ejercicio fue hecho correctamente, el conteo de ocurrencias de cada
nivel del factor será:

\begin{Shaded}
\begin{Highlighting}[]
\NormalTok{> }\KeywordTok{tabulate}\NormalTok{(usa3$Ing.Cat)}
\NormalTok{[}\DecValTok{1}\NormalTok{] }\DecValTok{12} \DecValTok{19} \DecValTok{11}  \DecValTok{1}
\end{Highlighting}
\end{Shaded}
Para que este factor sea más legible, se pueden cambiar los nombres de
los niveles:

\begin{Shaded}
\begin{Highlighting}[]
\KeywordTok{levels}\NormalTok{(usa3$Ing.Cat) <- }\KeywordTok{c}\NormalTok{(}\StringTok{"D"}\NormalTok{, }\StringTok{"C"}\NormalTok{, }\StringTok{"B"}\NormalTok{, }\StringTok{"A"}\NormalTok{)}
\end{Highlighting}
\end{Shaded}
En segundo lugar, se utilizará la función \texttt{tapply}, una variante
bastante especializada de \texttt{apply} (y muy similar a la función
\texttt{by}), para analizar los valores de analfabetismo
correspondientes a estas categorías de ingresos. La función
\texttt{tapply} se usa con la siguiente sintaxis:

\begin{Shaded}
\begin{Highlighting}[]
\KeywordTok{tapply}\NormalTok{(x, f, fu)}
\end{Highlighting}
\end{Shaded}
En dónde \texttt{x} es típicamente un vector numérico, \texttt{f} es un
factor cuya longitud equivale a la de \texttt{x} y \texttt{fu} es una
función de R (p.ej.: \texttt{mean}). Aquí lo que haría este comando es
ejecutar la función \texttt{fu} tantas veces como niveles tiene
\texttt{f} usando como entrada los elementos de \texttt{x} que se
corresponden con las ocurrencias de dicho nivel. Es decir, se ejecuta

\begin{Shaded}
\begin{Highlighting}[]
\KeywordTok{fu}\NormalTok{(x[f == }\KeywordTok{levels}\NormalTok{(f)[i]])}
\end{Highlighting}
\end{Shaded}
siendo \texttt{i} variable en \texttt{1:length(levels(f))}. Como
resultado devuelve una lista en la que cada elemento se corresponde con
la salida de uno de estos comandos.

Lo que usted deberá ejecutar aquí es la función \texttt{tapply} sobre el
vector numérico \texttt{Analf}, con \texttt{Ing.Cat} como factor de
referencia (ambas columnas de \texttt{usa3}) y la función
\texttt{summary}. El resultado, tal como se muestra en el archivo de
código fuente, debe guardarse en el objeto \texttt{salidaTapply}.

Finalmente debe hacer algo similar con la función \texttt{boxplot}, cuya
sintaxis es tal vez más sencilla que \texttt{tapply}, por ejemplo:

\begin{Shaded}
\begin{Highlighting}[]
\KeywordTok{boxplot}\NormalTok{(y ~ f, d)}
\end{Highlighting}
\end{Shaded}
Aquí \texttt{d} es una data.frame, mientras que \texttt{y} y \texttt{f}
son columnas de \texttt{d} de las clases ``numeric'' y ``factor''
respectivamente. Nuevamente las columnas a utilizar son \texttt{Analf} e
\texttt{Ing.Cat}, de la data.frame \texttt{usa3}. La salida de esta
función es doble, por un lado un objeto (el cual deberá guardar bajo el
nombre \texttt{salidaBoxplot}) y por otro una gráfica similar a la
Figura 1 de este repartido.

\begin{figure}[htbp]
\centering
\includegraphics{figure/unnamed-chunk-19.png}
\caption{Salida de boxplot}
\end{figure}

\subsubsection{1.g Exportar}

Script: ``exportar.R''

Finalmente se deberá exportar la data.frame \texttt{usaNorm} a un
archivo de texto plano. Dicho archivo se llamará ``usa-norm.csv'' y
deberá cumplir con las siguientes condiciones:

\begin{enumerate}[1.]
\item
  Deberá guardar correctamente los nombres de las filas y columnas.
\item
  El separador de columnas deberá ser el caracter \texttt{;}.
\item
  El punto decimal debe estar indicado con el caracter \texttt{,}.
\end{enumerate}
Consulte las lecciones o la ayuda de R en \texttt{?write.table} para
determinar el comando adecuado para realizar esta operación. En la
carpeta del repartido se incluye el archivo
\textbf{usa-norm-ejemplo.csv} para que usted pueda comparar con su
resultado (puede abrirlo con un programa de hojas de cálculo o con un
editor de texto plano, como bloc de notas o RStudio, para ver su
estructura interna). Nótese que este ejemplo sólo tiene 10 filas
elegidas al azar.

\end{document}
