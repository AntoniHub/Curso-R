\documentclass[]{article}
\hyphenation{co-rres-pon-dien-tes te-ner E-ben-sper-ger de-pre-da-do-res dis-po-ni-ble be-ne-fi-cio-sa in-di-vi-dual so-cia-li-dad mos-tra-ron fuen-tes a-cep-ta-ble ta-ma-ño o-pues-ta mo-de-lo es-tu-dian-tes e-jer-ci-cios co-rres-pon-dien-te mo-di-fi-ca-dos mo-di-fi-car-lo}
\usepackage{amssymb,amsmath}
\usepackage{ifxetex,ifluatex}
\ifxetex
  \usepackage{fontspec,xltxtra,xunicode}
  \defaultfontfeatures{Mapping=tex-text,Scale=MatchLowercase}
\else
  \ifluatex
    \usepackage{fontspec}
    \defaultfontfeatures{Mapping=tex-text,Scale=MatchLowercase}
  \else
    \usepackage[utf8]{inputenc}
  \fi
\fi
\usepackage{color}
\usepackage{fancyvrb}
\DefineShortVerb[commandchars=\\\{\}]{\|}
\DefineVerbatimEnvironment{Highlighting}{Verbatim}{commandchars=\\\{\}}
% Add ',fontsize=\small' for more characters per line
\newenvironment{Shaded}{}{}
\newcommand{\KeywordTok}[1]{\textcolor[rgb]{0.00,0.44,0.13}{\textbf{{#1}}}}
\newcommand{\DataTypeTok}[1]{\textcolor[rgb]{0.56,0.13,0.00}{{#1}}}
\newcommand{\DecValTok}[1]{\textcolor[rgb]{0.25,0.63,0.44}{{#1}}}
\newcommand{\BaseNTok}[1]{\textcolor[rgb]{0.25,0.63,0.44}{{#1}}}
\newcommand{\FloatTok}[1]{\textcolor[rgb]{0.25,0.63,0.44}{{#1}}}
\newcommand{\CharTok}[1]{\textcolor[rgb]{0.25,0.44,0.63}{{#1}}}
\newcommand{\StringTok}[1]{\textcolor[rgb]{0.25,0.44,0.63}{{#1}}}
\newcommand{\CommentTok}[1]{\textcolor[rgb]{0.38,0.63,0.69}{\textit{{#1}}}}
\newcommand{\OtherTok}[1]{\textcolor[rgb]{0.00,0.44,0.13}{{#1}}}
\newcommand{\AlertTok}[1]{\textcolor[rgb]{1.00,0.00,0.00}{\textbf{{#1}}}}
\newcommand{\FunctionTok}[1]{\textcolor[rgb]{0.02,0.16,0.49}{{#1}}}
\newcommand{\RegionMarkerTok}[1]{{#1}}
\newcommand{\ErrorTok}[1]{\textcolor[rgb]{1.00,0.00,0.00}{\textbf{{#1}}}}
\newcommand{\NormalTok}[1]{{#1}}
% Redefine labelwidth for lists; otherwise, the enumerate package will cause
% markers to extend beyond the left margin.
\makeatletter\AtBeginDocument{%
  \renewcommand{\@listi}
    {\setlength{\labelwidth}{4em}}
}\makeatother
\usepackage{enumerate}
\ifxetex
  \usepackage[setpagesize=false, % page size defined by xetex
              unicode=false, % unicode breaks when used with xetex
              xetex,
              colorlinks=true,
              linkcolor=blue]{hyperref}
\else
  \usepackage[unicode=true,
              colorlinks=true,
              linkcolor=blue]{hyperref}
\fi
\hypersetup{breaklinks=true, pdfborder={0 0 0}}
\setlength{\parindent}{0pt}
\setlength{\parskip}{6pt plus 2pt minus 1pt}
\setlength{\emergencystretch}{3em}  % prevent overfull lines
\setcounter{secnumdepth}{0}


\begin{document}

\section{Ejercicios de programación II: Fundamentos}

\subsubsection{{[}IMSER 2012{]}}

\begin{center}\rule{3in}{0.4pt}\end{center}

\subsection{Archivos incluidos:}

El
\href{http://eva.universidad.edu.uy/file.php/1454/ejercicios\_de\_programacion/rep-1.zip}{archivo}
con los ejercicios del práctico debe bajarse y descomprimirse en disco
duro, creando la carpeta \textbf{\texttt{rep-2}} (nota: no debe dentro
de ningún disco, partición o carpeta protegida a la escritura, como
puede ser un disco duro externo de backup). Usted deberá abrir el
RStudio y seleccionar dicha carpeta como su directorio de trabajo con
\texttt{setwd} o en RStudio la combinación \textbf{Ctrl + Shift + K}. En
esta carpeta se encuentran algunos archivos que usted deberá modificar:

\begin{itemize}
\item
  \textbf{\texttt{aprobados.R}}
\item
  \textbf{\texttt{aprobados2.R}}
\item
  \textbf{\texttt{mejorcitos.R}}
\item
  \textbf{\texttt{franjas.R}}
\item
  \textbf{\texttt{data.frame.R}}
\item
  \textbf{\texttt{ordenacion.R}}
\item
  \textbf{\texttt{lista.R}}
\item
  \textbf{\texttt{print.listaCalif.R}}
\end{itemize}
Adicionalmente los siguientes archivos son necesarios, pero \textbf{no
deben ser modificados} para que el método de calificación automático
funcione correctamente.

\begin{itemize}
\item
  \texttt{evaluar.R}
\item
  \texttt{notas.csv}
\item
  \texttt{datos}
\item
  \texttt{INSTRUCCIONES.pdf}
\item
  \texttt{calificaciones.R}
\item
  \texttt{ej2.rda}
\end{itemize}
\subsection{Mecanismo de corrección:}

Lo primero que debe hacer es cargar el archivo evaluar.R con la función
\texttt{source} y la codificación de caracteres ``UTF-8'' (lo cual
afecta a la función \texttt{evaluar} en particular), de la siguiente
manera:

\begin{Shaded}
\begin{Highlighting}[]
\KeywordTok{options}\NormalTok{(}\DataTypeTok{encoding =} \StringTok{"utf-8"}\NormalTok{)}
\KeywordTok{source}\NormalTok{(}\StringTok{"evaluar.R"}\NormalTok{)}
\end{Highlighting}
\end{Shaded}
Si usted ha ejecutado todos los pasos anteriores correctamente, la
siguiente frase debería verse en la consola:

\begin{verbatim}
Archivo de codigo fuente cargado correctamente
\end{verbatim}
En caso de que ocurra un error o se vea otro mensaje en la consola,
verifique que los archivos se descomprimieron correctamente y que usted
está trabajando en la carpeta correspondiente con el comando
\texttt{getwd()}.

Usted trabajará modificando los contenidos de dichos archivos con
RStudio (u otro programa de su preferencia) según las consignas que se
describen a continuación. Luego de terminar cada ejercicio y
\textbf{guardando el archivo} correspondiente en el disco duro, usted
podrá verificar rápidamente si su respuesta es correcta ejecutando el
comando:

\begin{Shaded}
\begin{Highlighting}[]
\KeywordTok{evaluar}\NormalTok{()}
\end{Highlighting}
\end{Shaded}
y además podrá en todo momento verificar su puntaje con la función
\texttt{verNotas()}. Tenga siempre en cuenta que, a \textbf{menos que
sea indicado} por la letra del ejercicio, las soluciones deben ser
genéricas y por lo tanto deben obtenerse con el código de los scripts en
lugar de ser valores fijos. Usualmente se utilizan valores generados de
forma aleatoria para las correcciones automáticas. Los objetos que son
evaluados en la corrección automática estarán indicados con un asterísco
en las instrucciones de cada script. Nótese además que en los archivos
\textbf{se indica claramente en dónde se inicia y dónde finaliza su
código} y que debe respetar esta organización para que la corrección de
los ejercicios funcione bien.

\subsubsection{Al finalizar}

Una vez terminados y guardados los archivos de los ejercicios del
repartido, usted deberá ejecutar \texttt{evaluar()} y seleccionar la
última opción (``Todos'') y luego subir el archivo ''datos'' (sin
extensión), incluido en la carpeta ''rep-1'', a la
\href{http://eva.universidad.edu.uy/mod/assignment/view.php?id=95125}{sección
de entregas} de la portada del curso en la plataforma EVA. Este archivo
se podrá reemplazar con uno más nuevo, en caso de que desee corregir
algún error; en caso de querer que el archivo sea corregido antes de la
fecha de entrega, puede cambiarle el nombre a ``datos-finalizado'', pero
en ese caso la nota no se cambiará de ahí en adelante.

\subsubsection{Código de Honor}

Si bien animamos a que los estudiantes trabajen en equipos y que haya un
intercambio fluido en los foros del curso, es fundamental que las
respuestas a los cuestionarios y ejercicios de programación sean fruto
del trabajo individual. En particular, consideramos importante que los
estudiantes no miren el código creado por sus compañeros ya que esto
supone un sabotaje a su propio proceso de aprendizaje. Como profesores
estamos comprometidos a pedir tareas para las cuales hayamos dado las
herramientas correctas y las explicaciones adecuadas como para que todos
puedan encontrar su propio camino para resolver los ejercicios.

\begin{center}\rule{3in}{0.4pt}\end{center}

\subsection{1. Datos de calificaciones}

Con el archivo ``calificaciones.R'' se generan calificaciones ficticias
de estudiantes de secundaria, utilizando rutinas de generación de
números aleatorios. Lo primero que debe hacer entonces es ejecutar el
comando

\begin{Shaded}
\begin{Highlighting}[]
\KeywordTok{source}\NormalTok{(}\StringTok{"calificaciones.R"}\NormalTok{)}
\end{Highlighting}
\end{Shaded}
para generar los vectores \texttt{cal} y \texttt{gen}, los cuales
deberían encontrarse en su área de trabajo (o ``Workspace'' en RStudio).
En el primero se encuentran las notas de dichos estudiantes, con valores
que van del 1 al 12, mientras que el segundo indica el sexo de cada uno
de ellos (codificados como ``V'' y ``M'', varón y mujer
respectivamente). Nótese que las clases de ambos vectores son
diferentes, siendo uno numérico y el otro caracter. Ejecute los
siguientes comandos para hechar un vistazo rápido a los datos:

\begin{Shaded}
\begin{Highlighting}[]
\CommentTok{# Histograma de las notas:}
\KeywordTok{hist}\NormalTok{(cal)}
\CommentTok{# Cantidad de mujeres y de varones:}
\KeywordTok{table}\NormalTok{(gen)}
\end{Highlighting}
\end{Shaded}
\subsubsection{1.a Porcentaje de aprobados}

Script: ``aprobados.R''

Considerando que los aprobados son todos aquellos que tienen una nota de
5 o mayor, usted deberá cuantificar el porcentaje de aprobaciones. En el
archivo \texttt{aprobados.R} deberá escribir el código necesario para
calcular este valor. Tenga en cuenta que para esta tarea necesitará usar
\href{http://eva.universidad.edu.uy/mod/resource/view.php?id=94745}{operadores
relacionales}. Existen varias formas de obtener el valor final, por lo
que queda a su gusto determinar el camino a tomar.

\subsubsection{1.b Aprobados por género}

Script: ``aprobados2.R''

El objetivo aquí es determinar la cantidad y porcentaje de aprobados
para varones y mujeres por separado. Es decir, se debe determinar la
cantidad de varones aprobados en relación a la cantidad de varones
totales, y lo mismo para las mujeres. El código para ejecutar esta tarea
se debe escribir en el archivo ``aprobados2.R''.

La estrategia más sencilla es primero separar las calificaciones de
varones y mujeres en dos vectores (\texttt{v} y \texttt{m} por ejemplo)
y luego aplicar el método usado en el ejercicio anterior en ambos. Para
esto evidentemente tendrá que usar el vector \texttt{gen} además de
operadores relacionales, en particular \texttt{==} (o su contraparte
\texttt{!=}).

\subsubsection{1.c Los mejorcitos}

Script: ``mejorcitos.R''

En esta parte lo que hará es seleccionar al grupo de los mejores
calificados, trabajando con el script ``mejorcitos.R''. El objetivo es
crear un vector numérico llamado \texttt{mejores}, el que tendrá las
calificaciones del 25\% de los estudiantes con mejor nota. Es decir, en
caso de que fueran 100 valores, queremos tomar los 25 más altos. Para
esto una estrategia simple es ordenar los valores de menor a mayor y
luego elegir los últimos elementos del vector resultante. Lo primero
será entonces crear un vector llamado \texttt{ord} con la función
\texttt{sort}. Posteriormente es necesario determinar la iésima posición
del mismo a partir de la cual se deben tomar los valores, para lo cual
seguirá la ecuación:

\[i = \lceil { n \cdot 0.75 } \rceil\]

donde $n$ es el número de elementos de \texttt{cal} y la función
$\lceil x \rceil$ indica la
\href{https://es.wikipedia.org/wiki/Funciones\_de\_parte\_entera\#Funci.C3.B3n\_techo}{función
techo}, es decir el siguiente valor entero más alto que $x$. Entonces el
vector \texttt{mejores} consistirá en los elementos de \texttt{ord}
entre las posiciones $i$ (inclusive) y $n$. Puede buscar la función
techo en R con el comando \texttt{??ceil}.

Finalmente puede visualizar cómo se distribuyen estos datos usando las
funciones \texttt{table} o \texttt{hist}.

\subsubsection{1.d Franjas de notas}

Script: ``franjas.R''

El objetivo de esta sección es crear un nuevo vector de clase
``character'' el cual tendrá las letras ``A'', ``B'', ``C'' y ``D'' a
fin de indicar cuatro franjas de notas, como se describe a continuación:

\begin{itemize}
\item
  A: $x \leq 3$
\item
  B: $3 < x \leq 6$
\item
  C: $6 < x \leq 9$
\item
  D: $9 < x$
\end{itemize}
siendo $x$ la nota del estudiante. El nuevo vector caracter se llamará
\texttt{ctg}, tendrá la misma cantidad de elementos que número de
calificaciones y para cada posición tendrá asignada la letra
correspondiente (en mayúsculas). Por ejemplo, para el siguiente vector
\texttt{cal}:

\begin{Shaded}
\begin{Highlighting}[]
\NormalTok{> cal}
\NormalTok{[}\DecValTok{1}\NormalTok{]  }\DecValTok{2} \DecValTok{11}  \DecValTok{3}  \DecValTok{6}  \DecValTok{7}  \DecValTok{6}  \DecValTok{9} \DecValTok{5}
\end{Highlighting}
\end{Shaded}
El vector \texttt{ctg} correspndiente es:

\begin{Shaded}
\begin{Highlighting}[]
\NormalTok{> ctg}
\NormalTok{[}\DecValTok{1}\NormalTok{] }\StringTok{"A"} \StringTok{"D"} \StringTok{"A"} \StringTok{"B"} \StringTok{"C"} \StringTok{"B"} \StringTok{"C"} \StringTok{"B"} 
\end{Highlighting}
\end{Shaded}
Finalmente deberá obtener la cantidad de casos para cada franja (en el
ejemplo anterior serían 2, 3, 2, 1 para A, B, C y D respectivamente) y
guardar estos valores en el vector \texttt{conteo} (las funciones
\texttt{sum} o \texttt{table} pueden ser de utilidad aquí). Si el vector
\texttt{conteo} resultante es de la clase ``table'' entonces debe
convertirlo en un vector ``numeric'' con la
\href{http://eva.universidad.edu.uy/file.php/1454/lecciones/2.6-identificadores-y-coercionadores.R}{función
coercionadora} adecuada y luego los nombres dicho vector deben cambiarse
a ``A'', ``B'', ``C'', y ``D'', utilizando la función \texttt{names}.

El archivo para esta tarea es ``franjas.R''.

\paragraph{Sugerencia:}

Una forma sencilla de hacer esto es crear un vector del tipo
``character'' inicial, con funciones tales como \texttt{rep} o
\texttt{character}, para luego modificarlo, utilizando el esquema:

\begin{Shaded}
\begin{Highlighting}[]
\NormalTok{x[i] <- y}
\end{Highlighting}
\end{Shaded}
en donde \texttt{x} es el vector caracter, \texttt{y} es un(os)
valor(es) determinado(s) e \texttt{i} es un vector (de uno o más
elementos) de clase ``integer'' o ``logical''. Probablemente deba usar
operadores
\href{http://eva.universidad.edu.uy/mod/resource/view.php?id=94745}{lógicos
y relacionales} para reproducir las 4 condiciones que definen las
franjas.

\subsection{2. Organización de los datos}

En esta sección se crearán data.frames y listas en base a datos de
calificaciones similares a los del ejercicio 1. Para esto puede
continuar utilizando los objetos creados en dicho ejercicio, o
alternativamente cargar objetos ya preparados para este ejercicio con el
comando:

\begin{Shaded}
\begin{Highlighting}[]
\CommentTok{# Comando opcional:}
\KeywordTok{load}\NormalTok{(}\StringTok{"ej2.rda"}\NormalTok{)}
\end{Highlighting}
\end{Shaded}
\subsubsection{2.a Crear una data.frame}

Sript: ``data.frame.R''

En esta sección deberá modificar el archivo ``data.frame.R'' para crear
un objeto de la clase ``data.frame'' llamado \texttt{datosCalif}, cuyas
columnas/variables sean los vectores \texttt{cal}, \texttt{gen} y
\texttt{ctg}. Los nombres de tales columnas en la data.frame serán
``nota'', ``genero'' y ``franja'' (respetando mayúsculas y minúsculas).
Utilice su método de preferencia para generar dicha data.frame. Una vez
hecho, el siguiente comando debería dar un resultado similar a este:

\begin{Shaded}
\begin{Highlighting}[]
\NormalTok{> }\KeywordTok{head}\NormalTok{(datosCalif)}
  \NormalTok{nota genero franja}
\DecValTok{1}    \DecValTok{9}      \NormalTok{M      C}
\DecValTok{2}    \DecValTok{8}      \NormalTok{V      C}
\DecValTok{3}    \DecValTok{5}      \NormalTok{M      B}
\DecValTok{4}   \DecValTok{11}      \NormalTok{V      D}
\DecValTok{5}   \DecValTok{11}      \NormalTok{V      D}
\DecValTok{6}    \DecValTok{5}      \NormalTok{M      B}
\end{Highlighting}
\end{Shaded}
\subsubsection{2.b Ordenar la tabla}

Script: ``ordenacion.R''

el objetivo aquí es modificar la tabla \texttt{datosCalif} creada en la
sección anterior, de forma que las filas estén ordenadas en función de
la nota (de menor a mayor). Una vez terminado, las primeras filas de
\texttt{datosCalif} se deberían ver similares a

\begin{Shaded}
\begin{Highlighting}[]
\NormalTok{> }\KeywordTok{head}\NormalTok{(datosCalif)}
    \NormalTok{nota genero franja}
\DecValTok{40}     \DecValTok{1}      \NormalTok{V      A}
\DecValTok{178}    \DecValTok{1}      \NormalTok{M      A}
\DecValTok{181}    \DecValTok{1}      \NormalTok{M      A}
\DecValTok{194}    \DecValTok{1}      \NormalTok{V      A}
\DecValTok{236}    \DecValTok{1}      \NormalTok{V      A}
\DecValTok{246}    \DecValTok{1}      \NormalTok{M      A}
\end{Highlighting}
\end{Shaded}
y las últimas parecidas a esto:

\begin{Shaded}
\begin{Highlighting}[]
\NormalTok{> }\KeywordTok{tail}\NormalTok{(datosCalif)}
    \NormalTok{nota genero franja}
\DecValTok{193}   \DecValTok{11}      \NormalTok{V      D}
\DecValTok{242}   \DecValTok{11}      \NormalTok{M      D}
\DecValTok{252}   \DecValTok{11}      \NormalTok{M      D}
\DecValTok{272}   \DecValTok{11}      \NormalTok{M      D}
\DecValTok{135}   \DecValTok{12}      \NormalTok{V      D}
\DecValTok{136}   \DecValTok{12}      \NormalTok{V      D}
\end{Highlighting}
\end{Shaded}
\paragraph{Sugerencias}

Para este ejercicio es necesario comprender el funcionamiento de la
función \texttt{order}, el uso de los índices en vectores y/o matrices,
así como el de los corchetes y/o el operador \texttt{\$}. Nótese además
que una expresión de la forma

\begin{Shaded}
\begin{Highlighting}[]
\NormalTok{x <- x[i]}
\end{Highlighting}
\end{Shaded}
es perfectamente válida y que el objeto \texttt{x} es sobreescrito en el
proceso.

\subsubsection{2.c Una lista con los datos}

Script: ``lista.R''

Vamos a crear ahora una lista con los datos generados (o los cargados
del archivo ``ej2.rda'' si es necesario), modificando el código del
archivo ``lista.R''. Esta lista se llamará \texttt{analisisCalif} y
tendrá los siguientes componentes (preste atención a los nombres):

\begin{enumerate}[1.]
\item
  tabla: la data.frame creada en el ejercicio 2.a (y modificada en 2.b).
\item
  conteo: el vector nombrado con los conteos por franjas de
  calificaciones.
\item
  aprob: una lista con 3 elementos:
  \begin{itemize}
  \item
    atot: porcentaje de aprobación total (\texttt{p.apr})
  \item
    avar: porcentaje de aprobación de varones (\texttt{p.apr.v})
  \item
    amuj: porcentaje de aprobación de mujeres (\texttt{p.apr.m})
  \end{itemize}
\end{enumerate}
Si su lista ha sido construida correctamente, al usar la función
\texttt{str} debería ver algo similar a lo siguiente:

\begin{Shaded}
\begin{Highlighting}[]
\NormalTok{> }\KeywordTok{str}\NormalTok{(analisisCalif)}
\NormalTok{List of }\DecValTok{3}
 \NormalTok{$ tabla :}\StringTok{'data.frame'}\NormalTok{:  }\DecValTok{272} \NormalTok{obs. of  }\DecValTok{3} \NormalTok{variables:}
  \NormalTok{..$ nota  : num [}\DecValTok{1}\NormalTok{:}\DecValTok{272}\NormalTok{] }\DecValTok{1} \DecValTok{1} \DecValTok{1} \DecValTok{1} \DecValTok{1} \DecValTok{1} \DecValTok{2} \DecValTok{2} \DecValTok{2} \DecValTok{2} \NormalTok{...}
  \NormalTok{..$ genero: Factor w/ }\DecValTok{2} \NormalTok{levels }\StringTok{"M"}\NormalTok{,}\StringTok{"V"}\NormalTok{: }\DecValTok{2} \DecValTok{1} \DecValTok{1} \DecValTok{2} \DecValTok{2} \DecValTok{1} \DecValTok{2} \DecValTok{1} \DecValTok{1} \DecValTok{1} \NormalTok{...}
  \NormalTok{..$ franja: Factor w/ }\DecValTok{4} \NormalTok{levels }\StringTok{"A"}\NormalTok{,}\StringTok{"B"}\NormalTok{,}\StringTok{"C"}\NormalTok{,}\StringTok{"D"}\NormalTok{: }\DecValTok{1} \DecValTok{1} \DecValTok{1} \DecValTok{1} \DecValTok{1} \DecValTok{1} \DecValTok{1} \DecValTok{1} \DecValTok{1} \DecValTok{1} \NormalTok{...}
 \NormalTok{$ conteo: Named num [}\DecValTok{1}\NormalTok{:}\DecValTok{4}\NormalTok{] }\DecValTok{38} \DecValTok{113} \DecValTok{99} \DecValTok{22}
  \NormalTok{..- }\KeywordTok{attr}\NormalTok{(*, }\StringTok{"names"}\NormalTok{)= chr [}\DecValTok{1}\NormalTok{:}\DecValTok{4}\NormalTok{] }\StringTok{"A"} \StringTok{"B"} \StringTok{"C"} \StringTok{"D"}
 \NormalTok{$ aprob :List of }\DecValTok{3}
  \NormalTok{..$ atot: num }\DecValTok{71}
  \NormalTok{..$ avar: num }\FloatTok{73.4}
  \NormalTok{..$ amuj: num }\FloatTok{68.2}
\end{Highlighting}
\end{Shaded}
\subsubsection{2.d Extra: una clase nueva y un método asociado}

(\emph{Este ejercicio es opcional, aunque puede sumar puntos en su
calificación final del repartido})

Script: ``print.listaCalif.R''

Cuando se manejan estructuras de información sofisticadas, como la lista
creada en la sección anterior, no es mala idea definir una nueva clase y
algunos métodos asociados para hacer el trabajo más fluido. Esto es útil
en particular cuando es una tarea que se repite muchas veces a lo largo
de un proyecto o en trabajos de rutina.

Actualmente el objeto \texttt{analisisCalif} debería ser de la clase
``list'', lo que podemos comprobar con el comando:

\begin{Shaded}
\begin{Highlighting}[]
\NormalTok{> }\KeywordTok{class}\NormalTok{(analisisCalif)}
\NormalTok{[}\DecValTok{1}\NormalTok{] }\StringTok{"list"}
\end{Highlighting}
\end{Shaded}
Hagamos una nueva clase llamada ``listaCalif'' la cual va a estar
compuestas por listas con la estructura de \texttt{analisisCalif}. Lo
primero que vamos a hacer es cambiar la clase de este objeto de la
siguiente manera:

\begin{Shaded}
\begin{Highlighting}[]
\NormalTok{> }\KeywordTok{class}\NormalTok{(analisisCalif) <- }\StringTok{"listaCalif"}
\end{Highlighting}
\end{Shaded}
Esto por ahora no traerá mayores cambios, pero una vez que usted
modifique el archivo ``print.listaCalif.R'' se habrá definido un nuevo
método de \texttt{print} para esta nueva clase. Para entender la
diferencia que esto hace, veamos lo que sucede cuando se escribe el
nombre del objeto \texttt{analisisCalif} en la consola:

\begin{Shaded}
\begin{Highlighting}[]
\NormalTok{> analisisCalif}
\KeywordTok{...}\NormalTok{(muchos números)}
\end{Highlighting}
\end{Shaded}
R simplemente arroja todos los datos de la lista. Sin embargo, al cargar
``print.listaCalif.R'', se puede ver algo así:

\begin{Shaded}
\begin{Highlighting}[]
\NormalTok{> }\KeywordTok{source}\NormalTok{(}\StringTok{"print.listaCalif.R"}\NormalTok{)}
\NormalTok{> analisisCalif}
\NormalTok{Porcentaje total de aprobaciones: }\FloatTok{70.96} \NormalTok
  \NormalTok{En mujeres: }\FloatTok{68.22} \NormalTok{%}
\NormalTok{La nota promedio fue de: }\OtherTok{NA} 
  \NormalTok{En varones: }\OtherTok{NA} 
  \NormalTok{En mujeres: }\OtherTok{NA} 
\NormalTok{Conteos por franja de nota:}
  \DecValTok{1}\ErrorTok{--}\DecValTok{3}   \DecValTok{4}\ErrorTok{--}\DecValTok{6}   \DecValTok{7}\ErrorTok{--}\DecValTok{9} \DecValTok{10}\ErrorTok{--}\DecValTok{12} 
    \DecValTok{38}    \DecValTok{113}     \DecValTok{99}     \DecValTok{22}
\end{Highlighting}
\end{Shaded}
¡Esta es la gran utilidad de definir métodos para una clase en R! En
este caso, en lugar de imprimir en pantalla una cantidad de números
difícles de leer, ahora se pueden ver los resultados más relevantes sin
todo ese ruido. Nótese que esto sólo es posible gracias a que
\texttt{print} es una función genérica y por lo tanto permite hacer
nuevos métodos en cualquier momento. No muchas funciones pertenecen a
esta categoría (ver \texttt{?GenericFunctions} por mayor información).

El único problema en este caso es que no se calculan bien los promedios
de las notas y por lo tanto se ven unos ``NA'' en la salida de la
consola. El objetivo de este ejercicio es modificar la función
\texttt{print.listaCalif} para que calcule estos 3 promedios y así los
imprima correctamente cada vez que se llama a un objeto de la nueva
clase ``listaCalif''. Invitamos además a mirar el resto del código de la
función \texttt{print.listaCalif} para tratar de entender los pasos que
se toman para generar esta salida.

Nótese el uso del operador \texttt{\$} en varios comandos internos de
dicha función, ya que usted deberá usarlos también para calcular los
promedios.

\end{document}
