\documentclass[]{article}
\usepackage{amssymb,amsmath}
\usepackage{ifxetex,ifluatex}
\ifxetex
  \usepackage{fontspec,xltxtra,xunicode}
  \defaultfontfeatures{Mapping=tex-text,Scale=MatchLowercase}
\else
  \ifluatex
    \usepackage{fontspec}
    \defaultfontfeatures{Mapping=tex-text,Scale=MatchLowercase}
  \else
    \usepackage[utf8]{inputenc}
  \fi
\fi
\usepackage{color}
\usepackage{fancyvrb}
\DefineShortVerb[commandchars=\\\{\}]{\|}
\DefineVerbatimEnvironment{Highlighting}{Verbatim}{commandchars=\\\{\}}
% Add ',fontsize=\small' for more characters per line
\newenvironment{Shaded}{}{}
\newcommand{\KeywordTok}[1]{\textcolor[rgb]{0.00,0.44,0.13}{\textbf{{#1}}}}
\newcommand{\DataTypeTok}[1]{\textcolor[rgb]{0.56,0.13,0.00}{{#1}}}
\newcommand{\DecValTok}[1]{\textcolor[rgb]{0.25,0.63,0.44}{{#1}}}
\newcommand{\BaseNTok}[1]{\textcolor[rgb]{0.25,0.63,0.44}{{#1}}}
\newcommand{\FloatTok}[1]{\textcolor[rgb]{0.25,0.63,0.44}{{#1}}}
\newcommand{\CharTok}[1]{\textcolor[rgb]{0.25,0.44,0.63}{{#1}}}
\newcommand{\StringTok}[1]{\textcolor[rgb]{0.25,0.44,0.63}{{#1}}}
\newcommand{\CommentTok}[1]{\textcolor[rgb]{0.38,0.63,0.69}{\textit{{#1}}}}
\newcommand{\OtherTok}[1]{\textcolor[rgb]{0.00,0.44,0.13}{{#1}}}
\newcommand{\AlertTok}[1]{\textcolor[rgb]{1.00,0.00,0.00}{\textbf{{#1}}}}
\newcommand{\FunctionTok}[1]{\textcolor[rgb]{0.02,0.16,0.49}{{#1}}}
\newcommand{\RegionMarkerTok}[1]{{#1}}
\newcommand{\ErrorTok}[1]{\textcolor[rgb]{1.00,0.00,0.00}{\textbf{{#1}}}}
\newcommand{\NormalTok}[1]{{#1}}
\ifxetex
  \usepackage[setpagesize=false, % page size defined by xetex
              unicode=false, % unicode breaks when used with xetex
              xetex,
              colorlinks=true,
              linkcolor=blue]{hyperref}
\else
  \usepackage[unicode=true,
              colorlinks=true,
              linkcolor=blue]{hyperref}
\fi
\hypersetup{breaklinks=true, pdfborder={0 0 0}}
\setlength{\parindent}{0pt}
\setlength{\parskip}{6pt plus 2pt minus 1pt}
\setlength{\emergencystretch}{3em}  % prevent overfull lines
\setcounter{secnumdepth}{0}


\begin{document}

\section{Parcial I}

\paragraph{Curso IMSER 2012}

\subsection{Instrucciones:}

La carpeta comprimida ``parcial-I.zip'' tiene un script de R
(``parcial.R'') en el cual usted deberá guardar todos los comandos del
ejercicio, siguiendo la demarcación que se muestra en el archivo mismo.

La hoja de cálculo llamada `datos.xls' contiene tres variables
muestreadas. La primer variable son los tamaños corporales (en Kg) de 40
competidores del último Mundial de Magic, 20 hombres y 20 mujeres. La
segunda variable es el sexo de los mismos, codificados como 1 para
mujeres y 2 para varones. La tercer variable es la altura de los
participantes, en metros.

Nota: los ejercicios del parcial son dependientes de los anteriores en
el sentido de que utilizan objetos creados, pero no implica que no se
puedan tratar de resolver independientemente.

Una vez terminado el parcial usted deberá comprimir la carpeta y subirla
al EVA en la
\href{http://eva.universidad.edu.uy/mod/assignment/view.php?id=99264}{página
correspondiente}. Nótese que debe contener al menos dos archivos:

\begin{itemize}
\item
  ``parcial.R''\\
\item
  Un archivo de texto plano (.txt o .csv) con los datos de
  ``datos.xls'', según lo indicado en la parte \textbf{a}.
\end{itemize}
\paragraph{a.}

Importar la tabla de la hoja de cálculo a R; el objeto resultante debe
ser una data.frame llamada \texttt{magic} y sus columnas deben llamarse
\texttt{body.size}, \texttt{sex} y \texttt{height} (i.e.: los valores
que R asigna por defecto).

Debe guardarse en la carpeta del parcial el archivo de texto (.txt o
.csv) utilizado para la importación.

\paragraph{b.}

Generar un vector aleatorio llamado \texttt{age}, compuesto por números
enteros positivos entre 18 y 35 aprox. (este rango puede tener cierta
flexibilidad). Nota: debe ser generado con una función creadora de
valores aleatorios, y se deben permitir ocurrencias repetidas del mismo
valor.

\paragraph{c.}

Agregar la variable \texttt{age} al data.frame \texttt{magic}.

\paragraph{d.}

La variable \texttt{sex} del data.frame presenta los valores 1 y 2.
Transformar esta variable en factor. Luego modificar los nombres de los
niveles del mismo a \texttt{"mujer"} y \texttt{"hombre"}
(correspondientes a los valores originales 1 y 2 respectivamente).

\paragraph{e.}

Crear la variable factorial \texttt{agef} basada en \texttt{age} de
forma tal que cuente con 3 niveles (dividiendo el rango en 3 franjas
etarias de igual amplitud). Agregar dicha variable a la data.frame
\texttt{magic}.

\paragraph{f.}

Nombrar a cada uno de los niveles de la variable \texttt{agef} (dentro
de \texttt{magic}) como \texttt{"novatos"}, \texttt{"intermedios"},
\texttt{"expertos"}, en ese orden.

\paragraph{g.}

Cambiar los nombres de las columnas del data.frame a \texttt{``peso''},
\texttt{``sexo''}, \texttt{``altura''}, \texttt{``edad''} y
\texttt{``edadf''}.

\paragraph{h.}

Graficar: peso en función de sexo. El gráfico debe ser un diagrama de
cajas y tener las etiquetas ``Sexo'', ``Peso (Kg)'' en los ejes
correspondientes, así como el título: ``Peso en función del sexo''.

\paragraph{i.}

Realizar un anova con la variable de respuesta \texttt{peso} y la
variable explicativa \texttt{sexo}; guardar el resultado en el objeto
\texttt{peso.sexo}.

\paragraph{j.}

Crear los objetos \texttt{peso.hombre} y \texttt{peso.mujer} con los
valores esperados de peso para los sexos respectivos, según los
resultados del modelo \texttt{peso.sexo} creado en el punto anterior
(considere la interpretación de los coeficientes del anova dada en la
lección correspondiente).

\paragraph{k.}

Graficar: $altura ^ 2$ (altura al cuadrado) en función del peso.

\paragraph{l.}

Realizar una regresión lineal entre estas dos variables ($altura ^ 2$
\ensuremath{\sim} peso) sin intercepto. El modelo obtenido debe
guardarse en el objeto \texttt{altura.peso}.

\paragraph{m.}

Realizar una regresión lineal entre estas dos variables ($altura ^ 2$
\ensuremath{\sim} peso) sin intercepto, pero esta vez excluyendo a los
outliers de peso (es decir, aquellos tales que peso \textgreater{} 120
Kg). Guardar el modelo en el objeto \texttt{altura.peso2}.

\paragraph{n.}

Utilizando los coeficientes obtenidos en este segundo modelo, determine
la altura esperada para la secuencia de pesos:

\begin{Shaded}
\begin{Highlighting}[]
\NormalTok{p <- }\KeywordTok{seq}\NormalTok{(}\DecValTok{40}\NormalTok{, }\DecValTok{120}\NormalTok{, }\DataTypeTok{by =} \FloatTok{0.5}\NormalTok{)}
\end{Highlighting}
\end{Shaded}
Los valores de altura esperados se deben guardar en un objeto llamado
\texttt{ae}.

(Nótese que el modelo establece la relación del peso con la
$altura ^ 2$, pero no con la altura per se).

\paragraph{o.}

Guardar en el objeto \texttt{r2} el valor del R cuadrado (no ajustado)
del modelo (es decir, el coeficiente de determinación o ``proporción de
varianza explicada'').

\paragraph{p.}

Agregar al último gráfico creado anteriormente las líneas
correspondientes a ambas regresiones lineales, utilizando diferentes
trazos y/o colores para diferenciarlas.

\paragraph{q.}

Agregar una línea vertical indicando el valor 120 en el eje del peso;
utilice un estilo de línea diferente a los anteriores (con color y/o
trazo diferente).

\paragraph{r.}

Superponer al mismo gráfico los puntos de los outliers en peso,
utilizando un símbolo diferente (y opcionalmente, un color distinto), de
forma tal que se puedan diferenciar a simple vista.

\paragraph{s.}

Agregar una última variable, llamada \texttt{IMC}, a la data.frame
\texttt{magic}: el
\href{http://es.wikipedia.org/wiki/\%C3\%8Dndice\_de\_masa\_corporal}{índice
IMC} correspondiente a cada participante, calculado como:

\[IMC = \frac{Peso (Kg)}{Altura ^ 2 (m)}\]

\paragraph{t.}

Exportar la data.frame \texttt{magic} a un archivo de texto plano (.txt
o .csv), conteniendo todas las columnas agregadas y los encabezados,
pero excluyendo los nombres de fila.

\end{document}
