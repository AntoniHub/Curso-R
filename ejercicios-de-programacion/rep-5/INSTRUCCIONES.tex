\documentclass[]{article}
\hyphenation{co-rres-pon-dien-tes te-ner E-ben-sper-ger de-pre-da-do-res dis-po-ni-ble be-ne-fi-cio-sa in-di-vi-dual so-cia-li-dad mos-tra-ron fuen-tes a-cep-ta-ble ta-ma-ño o-pues-ta mo-de-lo es-tu-dian-tes e-jer-ci-cios co-rres-pon-dien-te mo-di-fi-ca-dos mo-di-fi-car-lo ma-ni-pu-lar}
\usepackage{amssymb,amsmath}
\usepackage{ifxetex,ifluatex}
\ifxetex
  \usepackage{fontspec,xltxtra,xunicode}
  \defaultfontfeatures{Mapping=tex-text,Scale=MatchLowercase}
\else
  \ifluatex
    \usepackage{fontspec}
    \defaultfontfeatures{Mapping=tex-text,Scale=MatchLowercase}
  \else
    \usepackage[utf8]{inputenc}
  \fi
\fi
\usepackage{color}
\usepackage{fancyvrb}
\DefineShortVerb[commandchars=\\\{\}]{\|}
\DefineVerbatimEnvironment{Highlighting}{Verbatim}{commandchars=\\\{\}}
% Add ',fontsize=\small' for more characters per line
\newenvironment{Shaded}{}{}
\newcommand{\KeywordTok}[1]{\textcolor[rgb]{0.00,0.44,0.13}{\textbf{{#1}}}}
\newcommand{\DataTypeTok}[1]{\textcolor[rgb]{0.56,0.13,0.00}{{#1}}}
\newcommand{\DecValTok}[1]{\textcolor[rgb]{0.25,0.63,0.44}{{#1}}}
\newcommand{\BaseNTok}[1]{\textcolor[rgb]{0.25,0.63,0.44}{{#1}}}
\newcommand{\FloatTok}[1]{\textcolor[rgb]{0.25,0.63,0.44}{{#1}}}
\newcommand{\CharTok}[1]{\textcolor[rgb]{0.25,0.44,0.63}{{#1}}}
\newcommand{\StringTok}[1]{\textcolor[rgb]{0.25,0.44,0.63}{{#1}}}
\newcommand{\CommentTok}[1]{\textcolor[rgb]{0.38,0.63,0.69}{\textit{{#1}}}}
\newcommand{\OtherTok}[1]{\textcolor[rgb]{0.00,0.44,0.13}{{#1}}}
\newcommand{\AlertTok}[1]{\textcolor[rgb]{1.00,0.00,0.00}{\textbf{{#1}}}}
\newcommand{\FunctionTok}[1]{\textcolor[rgb]{0.02,0.16,0.49}{{#1}}}
\newcommand{\RegionMarkerTok}[1]{{#1}}
\newcommand{\ErrorTok}[1]{\textcolor[rgb]{1.00,0.00,0.00}{\textbf{{#1}}}}
\newcommand{\NormalTok}[1]{{#1}}
% Redefine labelwidth for lists; otherwise, the enumerate package will cause
% markers to extend beyond the left margin.
\makeatletter\AtBeginDocument{%
  \renewcommand{\@listi}
    {\setlength{\labelwidth}{4em}}
}\makeatother
\usepackage{enumerate}
\usepackage{graphicx}
% We will generate all images so they have a width \maxwidth. This means
% that they will get their normal width if they fit onto the page, but
% are scaled down if they would overflow the margins.
\makeatletter
\def\maxwidth{\ifdim\Gin@nat@width>\linewidth\linewidth
\else\Gin@nat@width\fi}
\makeatother
\let\Oldincludegraphics\includegraphics
\renewcommand{\includegraphics}[1]{\Oldincludegraphics[width=\maxwidth]{#1}}
\ifxetex
  \usepackage[setpagesize=false, % page size defined by xetex
              unicode=false, % unicode breaks when used with xetex
              xetex,
              colorlinks=true,
              linkcolor=blue]{hyperref}
\else
  \usepackage[unicode=true,
              colorlinks=true,
              linkcolor=blue]{hyperref}
\fi
\hypersetup{breaklinks=true, pdfborder={0 0 0}}
\setlength{\parindent}{0pt}
\setlength{\parskip}{6pt plus 2pt minus 1pt}
\setlength{\emergencystretch}{3em}  % prevent overfull lines
\setcounter{secnumdepth}{0}


\begin{document}

\section{Ejercicios de programación V: Funciones}

\subsubsection{{[}IMSER 2013{]}}

\begin{center}\rule{3in}{0.4pt}\end{center}

\subsection{Archivos incluidos:}

El archivo con los ejercicios del práctico debe bajarse y descomprimirse
en disco duro, creando la carpeta \textbf{\texttt{rep-X}} (nota: no debe
dentro de ningún disco, partición o carpeta protegida a la escritura,
como puede ser un disco duro externo de backup). Usted deberá abrir el
RStudio y seleccionar dicha carpeta como su directorio de trabajo con
\texttt{setwd} o en RStudio la combinación \textbf{Ctrl + Shift + K}. En
esta carpeta se encuentran algunos archivos que usted deberá modificar:

\begin{itemize}
\item
  \textbf{\texttt{1-triangulo.R}}
\item
  \textbf{\texttt{2.a-filtroc.R}}
\item
  \textbf{\texttt{2.b-extra-aplicar.R}}
\item
  \textbf{\texttt{2.c-educacion.R}}
\item
  \textbf{\texttt{3.a-cambia.pares.R}}
\item
  \textbf{\texttt{3.b-radio.R}}
\item
  \textbf{\texttt{3.c-extra-distancias.R}}
\end{itemize}
Adicionalmente los siguientes archivos son necesarios, pero \textbf{no
deben ser modificados} para que el método de calificación automático
funcione correctamente:

\begin{itemize}
\item
  \texttt{datos}
\item
  \texttt{evaluar.R}
\item
  \texttt{edu.data.RData}
\item
  \texttt{HandbookSpanish.pdf}
\item
  \texttt{INSTRUCCIONES.pdf}
\end{itemize}
\subsection{Mecanismo de corrección:}

Nota: más recomendaciones \textbf{importantes} se hacen en el documento
\href{http://goo.gl/P5Wnq}{Dinámica de los repartidos}.

Lo primero que debe hacer es cargar el archivo evaluar.R con la función
\texttt{source} y la codificación de caracteres ``UTF-8'' (lo cual
afecta a la función \texttt{evaluar} en particular), de la siguiente
manera:

\begin{Shaded}
\begin{Highlighting}[]
\KeywordTok{source}\NormalTok{(}\StringTok{"evaluar.R"}\NormalTok{, }\DataTypeTok{encoding =} \StringTok{"UTF-8"}\NormalTok{)}
\end{Highlighting}
\end{Shaded}
Nótese que hemos dejado de usar la función \texttt{options}, de forma
que de ahora en más \textbf{no ejecute el comando}:

\begin{Shaded}
\begin{Highlighting}[]
\KeywordTok{options}\NormalTok{(}\DataTypeTok{encoding =} \StringTok{"utf-8"}\NormalTok{)  }\CommentTok{# No me ejecuten!}
\end{Highlighting}
\end{Shaded}
Este cambio se debe a que hemos detectado que esta elección trae más
problemas que soluciones.

Si usted ha ejecutado todos los pasos anteriores correctamente, al usar
el comando \texttt{ls()} verá que \texttt{"evaluar"} figura en su sesión
y además en la consola debería ver lo siguiente:

\begin{verbatim}
Archivo de código fuente cargado correctamente

Chequeo de encoding:
  Los siguientes caracteres deben ser vocales con tilde:
    á - é - í - ó - ú
  Si *no se ven correctamente* corra el siguiente comando:
    source('evaluar.R', encoding = 'UTF-8')
\end{verbatim}
Usted trabajará modificando los contenidos de los archivos de los
ejercicios con RStudio (u otro programa de su preferencia) según las
consignas que se describen a continuación. Luego de terminar cada
ejercicio y \textbf{guardando el archivo} correspondiente en el disco
duro, usted podrá verificar rápidamente si su respuesta es correcta
ejecutando el comando:

\begin{Shaded}
\begin{Highlighting}[]
\KeywordTok{evaluar}\NormalTok{()}
\end{Highlighting}
\end{Shaded}
y además podrá en todo momento verificar su puntaje con la función
\texttt{verNotas()}. Tenga siempre en cuenta que, a \textbf{menos que
sea indicado} por la letra del ejercicio, las soluciones deben ser
genéricas y por lo tanto deben servir aún si se modifican los datos
originales (i.e.: no use valores fijos si no comandos). Usualmente se
utilizan valores generados de forma aleatoria para las correcciones
automáticas. Los objetos que son evaluados en la corrección automática
estarán indicados con un asterísco en las instrucciones de cada script.
Nótese además que en los archivos \textbf{se indica claramente en dónde
se inicia y dónde finaliza su código} y que debe respetar esta
organización para que la corrección de los ejercicios funcione bien.

\paragraph{NOTA:}

se agregó la función \texttt{fecha.datos} para facilitar el acceso al a
información de 1. cuál es la versión que usted tiene en su pc y 2. cómo
encontrar la fecha de la última versión.

\subsubsection{Al finalizar}

Una vez terminados y guardados los archivos de los ejercicios del
repartido, usted deberá ejecutar \texttt{evaluar()} y seleccionar la
última opción (``Todos'') y luego subir el archivo ''datos'' (sin
extensión), incluido en la carpeta ''rep-1'', a la
\href{http://eva.universidad.edu.uy/mod/assign/view.php?id=103966}{sección
de entregas} de la portada del curso en la plataforma EVA. Este archivo
se podrá reemplazar con uno más nuevo, en caso de que desee corregir
algún error; en caso de querer que el archivo sea corregido antes de la
fecha de entrega, puede cambiarle el nombre a ``datos-finalizado'', pero
en ese caso la nota no se cambiará de ahí en adelante.

\subsubsection{Código de Honor}

Si bien animamos a que trabaje en equipos y que haya un intercambio
fluido en los foros del curso, es fundamental que las respuestas a los
cuestionarios y ejercicios de programación sean fruto del trabajo
individual. En particular, consideramos necesario que no utilice el
código creado por sus compañeros, si no que debe programar sus propias
instrucciones, ya que de lo contrario supone un sabotaje a su propio
proceso de aprendizaje. Esto implica también evitar, en la medida de lo
posible, exponer el código propio a sus colegas. Como profesores estamos
comprometidos a dar nuestro mayor esfuerzo para dar las herramientas y
explicaciones adecuadas a fin de que pueda encontrar su propio camino
para resolver los ejercicios.

En casos de planteos de dudas a través del foro, en los que considere
que es imposible expresar un problema sin exponer su própio código,
entonces es aceptable hacerlo. De todas formas en estos casos es
preferible que envíe su código por correo electrónico directamente a un
profesor, explicando la problemática.

\begin{center}\rule{3in}{0.4pt}\end{center}

\subsection{1. Triángulos, volumen II}

Si hacemos un poco de memoria recordaremos que en el ejercicio 1 del
repartido I, tomábamos los catetos de un triángulo rectángulo y
escribíamos el código necesario para calcular el valor de la hipotenusa
y el área del mismo. Como vimos entonces, el valor de la hipotenusa se
calcula como:

\[
  hip = \sqrt{cat.op ^ 2 + cat.ad ^ 2}
\]

mientras que el área del triángulo rectángulo es:

\[
  A = \frac{cat.op \cdot cat.ad}{2}
\]

A partir de los valores de los catetos y de la hipotenusa es posible
calcular el valor de los restantes ángulos del triángulo, siendo el
ángulo adyacente

\[
  \alpha_{ad} = arccos \left( \frac{cat.ad}{hip} \right)
\]

y el ángulo opuesto

\[
  \alpha_{op} = arccos \left( \frac{cat.op}{hip} \right)
\]

En el presente ejercicio debemos escribir el código de una función
llamada \texttt{triangulo} que, a partir de los mismos argumentos de las
funciones \texttt{area} e \texttt{hipot} de aquella ocasión, o sea de
los catetos de un triángulo rectángulo, calcule la hipotenusa, el área y
los ángulos adyacente y opuesto del triángulo. La salida de la función
deberá ser una lista con los objetos denominados de la siguiente manera:
\texttt{hipotenusa}, \texttt{area}, \texttt{angulo.adyacente} y
\texttt{angulo.opuesto}, en ese orden. Los ángulos deberán estar
expresados en grados. En R, la salida de las funciones trigonométricas,
como \texttt{asin} y \texttt{acos} están expresadas en radianes, por lo
que se deberá hacer la transformación correspondiente, teniendo en
cuenta que $\pi$ radianes equivalen a $180^o$, y por lo tanto:

\[
  \alpha_{gr} = 180 \cdot \frac{\alpha_{rad}}{\pi}
\]

siendo $\alpha_{gr}$ y $\alpha_{rad}$ los equivalentes en grados y
radianes de un ángulo $\alpha$ cualquiera. A modo de ejemplo, la salida
deseada al evaluar esta función con el cateto adyacente y el opuesto
valiendo respectivamente 4 y 3, sería la siguiente:

\begin{Shaded}
\begin{Highlighting}[]
\KeywordTok{triangulo}\NormalTok{(}\DecValTok{4}\NormalTok{, }\DecValTok{3}\NormalTok{)}
\NormalTok{$hipotenusa}
\NormalTok{[}\DecValTok{1}\NormalTok{] }\DecValTok{5}

\NormalTok{$area}
\NormalTok{[}\DecValTok{1}\NormalTok{] }\DecValTok{6}

\NormalTok{$angulo.adyacente}
\NormalTok{[}\DecValTok{1}\NormalTok{] }\FloatTok{36.8699}

\NormalTok{$angulo.opuesto}
\NormalTok{[}\DecValTok{1}\NormalTok{] }\FloatTok{53.1301}
\end{Highlighting}
\end{Shaded}
(Recuerde que la función triángulo debe dar el resultado correcto para
cualquier par de valores positivos que sean asignados a los argumentos.)

\begin{center}\rule{3in}{0.4pt}\end{center}

\subsection{2. Educación}

La Organización de la Naciones Unidas (ONU) en el año 2011
\href{http://mdgs.un.org/unsd/mdg/Resources/Static/Products/Progress2011/11-31342\%20\%28S\%29\%20MDG\%20Report\%202011\_Book\%20LR.pdf}{presentó
un informe} basado en indicadores utilizados para medir y supervisar los
Objetivos de Desarrollo del Milenio (ODM).

\begin{quote}
``La Declaración del Milenio de las Naciones Unidas de 2000, basada en
las conferencias mundiales de las Naciones Unidas durante el decenio de
1990, representó un fuerte compromiso con el derecho al desarrollo, la
paz y la seguridad, la igualdad de género, la erradicación de las
numerosas dimensiones de la pobreza y el desarrollo humano sostenible.
En la Declaración, adoptada por 147 jefes de Estado y 189 Estados, se
incorporaban lo que ha llegado a conocerse con el nombre de `ocho
objetivos de desarrollo del milenio', incluidas 18 metas con plazos
cronológicos delimitados''
(\href{http://unstats.un.org/unsd/publication/seriesf/Seriesf\_95s.pdf}{Fuente}).

\end{quote}
Para este ejercicio se utilizarán tres de los indicadores que han sido
propuestos para cumplir con el objetivo 2 de los ODM
\href{http://www.undp.org/content/undp/es/home/mdgoverview/mdg\_goals/mdg2/}{Lograr
la enseñanza primaria universal}:

\begin{itemize}
\item
  Tasa Neta de Matriculación en la Enseñanza Primaria (\texttt{TM}).
\item
  Porcentaje de alumnos que comienzan el primer grado y llegan al quinto
  grado (\texttt{PA}).
\item
  Tasa de alfabetización de personas entre 15 y 24 años (\texttt{TA}).
\end{itemize}
Las tres variables están expresadas en porcentaje. El objetivo es
realizar un cálculo grueso del porcentaje de niños que completan la
educación primaria (\texttt{PC}) para ciertos paises, a partir de
\texttt{TM} y \texttt{PA}, usando la fórmula:
$PC = (TM \cdot PA) / 100$. Finalmente se hará una regresión entre la
tasa de alfabetización y el porcentaje de finalización de primaria
calculado (i.e.: \texttt{TA} en función de \texttt{PC}).

El archivo ``edu.data.RData'' contiene una data.frame llamada
\texttt{edu.data} con datos de varios años para ciertos países, usando
las abreviaciones de arriba para cada indicador (importar a R con
\texttt{load} o los botones de RStudio). Los datos fueron extraidos de
la \href{http://mdgs.un.org/unsd/mdg/Data.aspx}{portada oficial} de la
ONU para los indicadores de los OMD.

Sin embargo no vamos a ir directamente a este cálculo, si no a
desarrollar funciones que permitan hacerlo rápidamente con cualquier
tabla de datos con una estructura similar a la que usamos como ejemplo.

\subsubsection{2.a Filtro por columnas}

Ya vimos que la función subset sirve para filtrar observaciones, las
filas, en un data.frame con el fin de eliminar aquellas que no sirven a
nuestros propósitos. En este ejercicio usted hará una función que haga
algo similar, pero con las columnas. Considere la función \texttt{grep}.
Esta sirve para reconocer patrones en vectores character, es decir,
puede identificar palabras en un texto. Por ejemplo:

\begin{Shaded}
\begin{Highlighting}[]
\NormalTok{v <- }\KeywordTok{c}\NormalTok{(}\StringTok{"glucosa"}\NormalTok{, }\StringTok{"citosina"}\NormalTok{, }\StringTok{"timina"}\NormalTok{, }\StringTok{"ribosa"}\NormalTok{, }\StringTok{"adenina"}\NormalTok{, }\StringTok{"sacarosa"}\NormalTok{, }\StringTok{"guanina"}\NormalTok{)}
\NormalTok{i <- }\KeywordTok{grep}\NormalTok{(}\StringTok{"osa"}\NormalTok{, v)}
\NormalTok{v[i]}
\end{Highlighting}
\end{Shaded}
\begin{verbatim}
## [1] "glucosa"  "ribosa"   "sacarosa"
\end{verbatim}
Como se puede ver aquí, \texttt{grep} devuelve un vector numérico con
las posiciones de \texttt{v} tales que el string ``osa'' es parte de la
palabra (en este casos, nombres de monosacáridos). Esto por supuesto
puede usarse con nombres de columnas y es lo que se pide que usted haga
en este ejercicio.

Vamos a usar esta capacidad para filtrar data.frames según sus nombres
de columnas. En el script de este ejercicio, usted deberá codificar la
función \texttt{filtroc}.

\paragraph{Argumentos:}

\begin{itemize}
\item
  \texttt{x}: una data.frame con nombres de columnas útiles para hacer
  el filtrado.
\item
  \texttt{clave}: string utilizado para seleccionar columnas (i.e.: un
  vector character de longitud 1). \textbf{Importante}: no debe importar
  si la \texttt{clave} está en mayúsculas o minúsculas (i.e.: da lo
  mismo usar \texttt{"TM"}, \texttt{"tm"} o \texttt{"tM"}). Para esto
  puede usar las funciones \texttt{tolower} o \texttt{toupper} para
  manipular vectores character.
\end{itemize}
\paragraph{Acciones:}

\begin{enumerate}[1.]
\item
  Con alguna de las funciones \texttt{grep} determinar cuáles son las
  columnas de \texttt{x} que contienen el string \texttt{clave}.
\item
  Devolver un data.frame con las columnas seleccionadas en punto
  anterior. Esto puede ser un problema cuando se selecciona una sola
  columna (e.g.: la columna \texttt{TA} en \texttt{edu.data}). Para
  evitarlos indice la data.frame \texttt{x} sin usar coma (vea la
  diferencia entre escribir \texttt{edu.data{[},11{]}} y
  \texttt{edu.data{[}11{]}} como ejemplo ilustrador).
\end{enumerate}
El siguiente ejemplo debería ilustrar mejor el funcionamiento esperado
de \texttt{filtroc}:

\begin{Shaded}
\begin{Highlighting}[]
\KeywordTok{load}\NormalTok{(}\StringTok{"edu.data.RData"}\NormalTok{)}
\KeywordTok{filtroc}\NormalTok{(edu.data, }\StringTok{"Tm"}\NormalTok{)  }\CommentTok{# Se usan mayúsculas y minúsculas}
\end{Highlighting}
\end{Shaded}
\begin{verbatim}
##                   TM_2005 TM_2006 TM_2007 TM_2008 TM_2009
## Aruba                97.5    98.3    98.7    97.5    94.7
## Colombia             96.3    95.3    94.2    93.7    93.2
## Cuba                 95.2    96.1    98.8    99.5    99.8
## El Salvador          95.1    95.0    94.7    94.6    94.6
## Guatemala            95.1    96.4    97.2    96.8      NA
## México               99.6    99.6    99.4    99.5    99.6
## Nicaragua            94.6    92.9    93.1    94.4      NA
## Panamá               99.1    99.0    99.0    98.9    97.1
## Paraguay             95.1    93.2    90.8    88.2    85.7
## Perú                 99.6    99.7    99.7    98.2    97.2
## Trinidad y Tobago    91.3      NA    96.9    95.0    96.0
## Uruguay              97.6    99.9      NA    99.3    99.5
## Venezuela            92.8    93.4    94.4      NA    94.2
\end{verbatim}
Nótese que esto permite elegir columnas con datos de \texttt{TM} sin
importar el año en que fueron tomados.

\paragraph{Sugerencia:}

Antes de utilizar \texttt{grep} convertir las letras de la
\texttt{clave} a mayúsculas/minúsculas según su preferencia; lo mismo
para un vector con los nombres de columnas de \texttt{x} (pero no para
los nombres de la data.frame en sí). Luego utilice estos vectores como
entrada de \texttt{grep}.

\subsubsection{2.b Extra: aplicar}

El objetivo es hacer una función que combine las funciones
\texttt{filtroc} (ej. 2.a) y \texttt{apply}. Esta función se llamará
\texttt{aplicar}. Durante la corrección el sistema utilizará su propia
versión de \texttt{filtroc}.

\paragraph{Argumentos:}

\begin{itemize}
\item
  \texttt{x}: una data.frame cualquiera.
\item
  \texttt{clave}: clave para seleccionar columnas (ver \texttt{filtroc},
  ej. 2.a).
\item
  \texttt{FUN}: función a aplicar a las filas de los datos de
  \texttt{x}, una vez que se seleccionan las columnas.
\item
  \texttt{...}: argumento especial, utilizado para enviar argumentos
  nombrados a la función \texttt{FUN}, cualquiera sea esta (puede ser
  una cantidad arbitraria de argumentos; \texttt{?dotsMethods} o el
  \href{http://cran.r-project.org/doc/manuals/r-release/R-intro.html\#The-three-dots-argument}{el
  manual introductorio} para ver la ayuda de R respecto a este argumento
  especial).
\end{itemize}
\paragraph{Acciones:}

\begin{enumerate}[1.]
\item
  Seleccionará las columnas de \texttt{x} según la \texttt{clave}
  dispuesta por el usuario (tal como lo debería hacer la función
  \texttt{filtro}), resultando en un data.frame.
\item
  Aplicará la función \texttt{FUN} a todas las \emph{filas} de la
  data.frame resultante del paso anterior (se sugiere usar
  \texttt{apply} aquí).
\item
  Devolverá un vector con los valores obtenidos en el paso anterior.
\end{enumerate}
\begin{Shaded}
\begin{Highlighting}[]
\KeywordTok{load}\NormalTok{(}\StringTok{"edu.data.RData"}\NormalTok{)}
\KeywordTok{aplicar}\NormalTok{(edu.data, }\StringTok{"pA"}\NormalTok{, mean, }\DataTypeTok{na.rm =} \OtherTok{TRUE}\NormalTok{)}
\end{Highlighting}
\end{Shaded}
\begin{verbatim}
##             Aruba          Colombia              Cuba       El Salvador 
##             90.43             83.50             96.30             74.90 
##         Guatemala            México         Nicaragua            Panamá 
##             63.53             92.82             47.50             87.84 
##          Paraguay              Perú Trinidad y Tobago           Uruguay 
##             78.15             87.42             91.40             93.88 
##         Venezuela 
##             90.36
\end{verbatim}
Nota: en este ejemplo el argumento especial \texttt{...} sirve para
enviar el argumento nombrado \texttt{na.rm = TRUE} a la función
\texttt{mean} (a través del argumento \texttt{...} de la propia función
\texttt{apply}). Los demás argumentos son: \texttt{x = edu.data},
\texttt{clave = "Tm"} y \texttt{FUN = mean}. En la corrección automática
se utilizarán otras funciones y argumentos para poner a prueba el
ejercicio.

\subsubsection{2.c Función educación}

La función \texttt{educacion} será la que calcule el \texttt{PC}
(``porcentaje de colnclusión'' o de ``finalización'' de la primaria) y
además haga la regresión entre \texttt{TA} y \texttt{PC}. La función
supone de antemano que todas las columnas tienen nombres con los
prefijos \texttt{TM}, \texttt{PA} o \texttt{TA}, y además que hay una
única columna \texttt{TA}. Si a usted le interesa puede agregar más
argumentos para que su función no parta de este supuesto, pero es una
opción no contemplada por el ejercicio.

Note que en este ejercicio es posible utilizar las funciones que se
piden en 2.a y 2.b, si bien no es obligatorio. De todas formas, durante
la corrección el sistema usará su propia versión de dichas funciones.

\paragraph{Argumentos:}

\begin{itemize}
\item
  \texttt{x}: data.frame con los datos de entrada (como
  \texttt{edu.data}), en la que deben estar los TM, PA y TA, tal como se
  explica anteriormente.
\end{itemize}
\paragraph{Acciones:}

\begin{enumerate}[1.]
\item
  Calcular, por país/fila, el promedio de los valores de TM y de PA.
  Para esto pueden ser útiles las funciones \texttt{filtroc} del
  ejercicio 2.a y \texttt{aplicar} del 2.b.
\item
  Calcular con estos el valor de PC para cada país (lo que debe hacerse
  con los promedios de TM y PA obtenidos en el punto anterior).
\item
  Agregar estos tres valores calculados como columnas a la data.frame de
  entrada (\texttt{x}).
\item
  Realizar una regresión lineal de TA en función de PC.
\item
  Devolver un objeto de clase list con 1. la regresión y 2. la
  data.frame modificada
\end{enumerate}
Nota: los cálculos de promedios deben hacerse dejando de lado los
valores \texttt{NA}.

\paragraph{Salida:}

\texttt{educacion} debe devolver un objeto de clase \texttt{list}, el
cual debe contener los siguientes objetos (y en el mismo orden):

\begin{itemize}
\item
  \texttt{reg}: el objeto \texttt{lm} obtenido al hacer la regresión
  lineal.
\item
  \texttt{datos}: data.frame con los datos originales más las columnas
  correspondientes a los promedios de \texttt{TM}, \texttt{PA} y
  \texttt{PC}, en ese orden.
\end{itemize}
Nota: la función \texttt{summary} además de imprimir en la consola datos
útiles de una regresión necesarios para este ejercicio, también genera
una \textbf{salida invisible} (ver lección 5.3).

El siguiente es un ejemplo obtenido con los datos y una función
\texttt{educacion} completada:

\begin{Shaded}
\begin{Highlighting}[]
\KeywordTok{load}\NormalTok{(}\StringTok{"edu.data.RData"}\NormalTok{)}
\NormalTok{e <- }\KeywordTok{educacion}\NormalTok{(edu.data)}
\NormalTok{e$reg}
\end{Highlighting}
\end{Shaded}
\begin{verbatim}
## 
## Call:
## lm(formula = TA ~ PC, data = x)
## 
## Coefficients:
## (Intercept)           PC  
##      75.095        0.268
\end{verbatim}
\begin{Shaded}
\begin{Highlighting}[]
\KeywordTok{str}\NormalTok{(e$datos)}
\end{Highlighting}
\end{Shaded}
\begin{verbatim}
## 'data.frame':    13 obs. of  14 variables:
##  $ TM_2005: num  97.5 96.3 95.2 95.1 95.1 99.6 94.6 99.1 95.1 99.6 ...
##  $ TM_2006: num  98.3 95.3 96.1 95 96.4 99.6 92.9 99 93.2 99.7 ...
##  $ TM_2007: num  98.7 94.2 98.8 94.7 97.2 99.4 93.1 99 90.8 99.7 ...
##  $ TM_2008: num  97.5 93.7 99.5 94.6 96.8 99.5 94.4 98.9 88.2 98.2 ...
##  $ TM_2009: num  94.7 93.2 99.8 94.6 NA 99.6 NA 97.1 85.7 97.2 ...
##  $ PA_2005: num  NA 80.8 96.9 67.3 63.3 92.1 50.2 85.2 76.2 NA ...
##  $ PA_2006: num  93.4 85.2 96.8 68.9 62.5 92.3 43.9 88.2 79.1 89.7 ...
##  $ PA_2007: num  86.5 NA 96.1 75.7 64.8 91.5 48.4 85.2 79.2 82 ...
##  $ PA_2008: num  91.4 NA 95.5 76.1 NA 94 NA 86.8 78.1 87.6 ...
##  $ PA_2009: num  NA 84.5 96.2 86.5 NA 94.2 NA 93.8 NA 90.4 ...
##  $ TA     : num  99.1 98 100 94.9 87 ...
##  $ TM     : num  97.3 94.5 97.9 94.8 96.4 ...
##  $ PA     : num  90.4 83.5 96.3 74.9 63.5 ...
##  $ PC     : num  88 78.9 94.3 71 61.2 ...
\end{verbatim}
\begin{center}\rule{3in}{0.4pt}\end{center}

\subsection{3. Funciones con problemas}

En este ejercicio se propone arreglar el código de algunas funciones
simples (y otra no tan simple). Para esto la idea es utilizar los
métodos y conceptos vertidos en la lección 5.4: \textbf{Depuración de
funciones}.

Recuerde que hay errores más comunes que otros. Por ejemplo funciones,
operadores u objetos \textbf{mal escritos}, \textbf{paréntesis},
\textbf{llaves}, o \textbf{corchetes} mal cerrados y \textbf{errores de
indización} (como poner 1 sólo número para indizar una matriz), son de
los más frecuentes. Otros también ``populares'' son el utilizar de forma
inconsistente las comillas (i.e.: mezclar \texttt{"} con \texttt{'}
descuidadamente) u olvidar poner una coma entre argumentos. Recomendamos
tengan cerca la lección 1.3 extra: \textbf{errores y afines}.

\subsubsection{3.a Cambiador de valores en subíndices pares\ldots{}}

Esta sencilla función toma un vector \texttt{x} y cambia aquellos
valores que se encuentran en las posiciones pares del mismo
(\texttt{x{[}2{]}}, \texttt{x{[}4{]}}, etc.), usando como sustitutos los
elementos del vector \texttt{subs} (el segundo argumento). Un ejemplo de
salida de esta función puede ser el siguiente:

\begin{Shaded}
\begin{Highlighting}[]
\KeywordTok{cambia.pares}\NormalTok{(}\DecValTok{1}\NormalTok{:}\DecValTok{6}\NormalTok{, }\OtherTok{NA}\NormalTok{)}
\end{Highlighting}
\end{Shaded}
\begin{verbatim}
## [1]  1 NA  3 NA  5 NA
\end{verbatim}
El objetivo de este ejercicio es arreglar el código de la función
contenida en el script asociado de forma tal que ejecute correctamente
su tarea. El código tiene cuatro errores,

\subsubsection{3.b Radios}

La función radio toma como argumento el valor \texttt{r}, un número
cualquiera, y calcula tres valores asociados con circunferencias y
esferas: el perímetro (P), área (A; de la circunferencia) y volumen (V;
de la esfera), utilizando las fórmulas:

\[
  P = 2 \cdot \pi \cdot r
\] \[
  A = \pi \cdot r^2
\] \[
  V = \frac{4 \cdot \pi \cdot r^3}{3}
\]

La función está pensada para generar una salida invisible, al mismo
tiempo que imprimir en la consola los resultados obtenidos (nótese el
uso de la función \texttt{cat} para este cometido). El siguiente es un
ejemplo de salida de la función:

\begin{Shaded}
\begin{Highlighting}[]
\NormalTok{x <- }\KeywordTok{radio}\NormalTok{(}\DecValTok{5}\NormalTok{)}
\end{Highlighting}
\end{Shaded}
\begin{verbatim}
## Perímetro: 31.42 
## Área:      78.54 
## Volumen:   523.6
\end{verbatim}
\begin{Shaded}
\begin{Highlighting}[]
\NormalTok{x}
\end{Highlighting}
\end{Shaded}
\begin{verbatim}
##      P      A      V 
##  31.42  78.54 523.60
\end{verbatim}
El objetivo de este ejercicio es arreglar el código de la función
contenida en el script asociado de forma tal que ejecute correctamente
su tarea. En total hay 5 errores (aunque 1 de ellos incluye 2 cambios en
una misma línea).

\subsubsection{3.c Extra: encuentra distancias}

La función \texttt{distancias}, escrita en el archivo ``distancias.R'',
busca realizar una tarea parecida a la que ya se hizo en el Repartido I
del curso: calcular las distancias de un punto a un conjunto de
coordenadas (ej. 1.c). En este caso se utiliza un
\href{http://www.johndcook.com/blog/2010/06/02/whats-so-hard-about-finding-a-hypotenuse/}{algoritmo
alternativo} para calcular las distancias, el cual es más robusto dadas
las limitaciones de las computadoras.

El significado de los argumentos de la función están explicados en el
propio archivo, así como varios de los pasos internos, a través del uso
de comentarios. Al igual que en ejercicios anteriores, el objetivo es
corregir los errores que tiene el archivo para que la función
\texttt{distancias} cumpla su tarea correctamente. El siguiente es un
ejemplo de cómo debería ser la salida de la función (incluyendo la
figura):

\begin{Shaded}
\begin{Highlighting}[]
\NormalTok{pts <- }\KeywordTok{matrix}\NormalTok{(}\KeywordTok{rnorm}\NormalTok{(}\DecValTok{20}\NormalTok{), }\DataTypeTok{ncol =} \DecValTok{2}\NormalTok{)}
\NormalTok{x <- }\KeywordTok{distancias}\NormalTok{(pts, }\DataTypeTok{p =} \KeywordTok{c}\NormalTok{(}\FloatTok{0.3}\NormalTok{, -}\FloatTok{0.1}\NormalTok{))}
\end{Highlighting}
\end{Shaded}
\begin{figure}[htbp]
\centering
\includegraphics{figure/unnamed-chunk-17.png}
\caption{salida gráfica de la función distancias}
\end{figure}

\begin{verbatim}
## d.max = 2.69 - punto: 0.86 2.53 
## d.min = 0.56 - punto: 0.63 0.35
\end{verbatim}
\begin{Shaded}
\begin{Highlighting}[]
\NormalTok{x}
\end{Highlighting}
\end{Shaded}
\begin{verbatim}
## $dists
## d.max d.min 
## 2.689 0.562 
## 
## $posiciones
## i.max i.min 
##     6     7 
## 
## $puntos
##            x      y
## d.max 0.8578 2.5306
## d.min 0.6330 0.3527
## 
## $centro
## [1]  0.3 -0.1
\end{verbatim}
Hay 5 errores para corregir, pero dos de ellos no van a generar ningún
mensaje de error ni de aviso, excepto al usar \texttt{evaluar}.

\end{document}
