\documentclass[]{article}
\usepackage{amssymb,amsmath}
\usepackage{ifxetex,ifluatex}
\ifxetex
  \usepackage{fontspec,xltxtra,xunicode}
  \defaultfontfeatures{Mapping=tex-text,Scale=MatchLowercase}
\else
  \ifluatex
    \usepackage{fontspec}
    \defaultfontfeatures{Mapping=tex-text,Scale=MatchLowercase}
  \else
    \usepackage[utf8]{inputenc}
  \fi
\fi
\usepackage{color}
\usepackage{fancyvrb}
\DefineShortVerb[commandchars=\\\{\}]{\|}
\DefineVerbatimEnvironment{Highlighting}{Verbatim}{commandchars=\\\{\}}
% Add ',fontsize=\small' for more characters per line
\newenvironment{Shaded}{}{}
\newcommand{\KeywordTok}[1]{\textcolor[rgb]{0.00,0.44,0.13}{\textbf{{#1}}}}
\newcommand{\DataTypeTok}[1]{\textcolor[rgb]{0.56,0.13,0.00}{{#1}}}
\newcommand{\DecValTok}[1]{\textcolor[rgb]{0.25,0.63,0.44}{{#1}}}
\newcommand{\BaseNTok}[1]{\textcolor[rgb]{0.25,0.63,0.44}{{#1}}}
\newcommand{\FloatTok}[1]{\textcolor[rgb]{0.25,0.63,0.44}{{#1}}}
\newcommand{\CharTok}[1]{\textcolor[rgb]{0.25,0.44,0.63}{{#1}}}
\newcommand{\StringTok}[1]{\textcolor[rgb]{0.25,0.44,0.63}{{#1}}}
\newcommand{\CommentTok}[1]{\textcolor[rgb]{0.38,0.63,0.69}{\textit{{#1}}}}
\newcommand{\OtherTok}[1]{\textcolor[rgb]{0.00,0.44,0.13}{{#1}}}
\newcommand{\AlertTok}[1]{\textcolor[rgb]{1.00,0.00,0.00}{\textbf{{#1}}}}
\newcommand{\FunctionTok}[1]{\textcolor[rgb]{0.02,0.16,0.49}{{#1}}}
\newcommand{\RegionMarkerTok}[1]{{#1}}
\newcommand{\ErrorTok}[1]{\textcolor[rgb]{1.00,0.00,0.00}{\textbf{{#1}}}}
\newcommand{\NormalTok}[1]{{#1}}
\ifxetex
  \usepackage[setpagesize=false, % page size defined by xetex
              unicode=false, % unicode breaks when used with xetex
              xetex,
              colorlinks=true,
              linkcolor=blue]{hyperref}
\else
  \usepackage[unicode=true,
              colorlinks=true,
              linkcolor=blue]{hyperref}
\fi
\hypersetup{breaklinks=true, pdfborder={0 0 0}}
\setlength{\parindent}{0pt}
\setlength{\parskip}{6pt plus 2pt minus 1pt}
\setlength{\emergencystretch}{3em}  % prevent overfull lines
\setcounter{secnumdepth}{0}


\begin{document}

\section{Dinámica de trabajo de los repartidos}

A lo largo del curso y en cada unidad, usted deberá completar repartidos
con los denominados \textbf{ejercicios de programación}. En estos se
pide al estudiante que complete un archivo de texto plano (un script de
R, y por lo tanto con la extensión \texttt{.R}) con el código (i.e.:
instrucciones en lenguaje R) necesarias para realizar las tareas
asignadas.

Para cada ejercicio usted podrá usar un sistema de corrección
automático, desarrollado para este curso, el cual le permitirá saber de
forma inmediata si ha completado correctamente las instrucciones del
mismo. Para que este método funcione correctamente, hay que seguir
ciertos procedimientos generales que pasaremos a describir a
continuación.

Este documento puede considerarse como una contraparte escrita al video
``Guía para repartidos''.

ADVERTENCIA: leer este documento puede ahorrar \emph{muchos} dolores de
cabeza a la hora de resolver los ejercicios del curso.

\subsection{Archivos incluidos:}

Cada repartido consta de una carpeta con archivos que usted puede bajar
desde la plataforma del curso (por ejemplo
\href{http://eva.universidad.edu.uy/mod/resource/view.php?id=93598}{este}).
Aquí por supuesto se encuentran los scripts de R que usted debe
\textbf{modificar}, así como ciertos archivos auxiliares que usted debe
\textbf{dejar como están} (usualmente: \texttt{evaluar.R},
\texttt{datos}, \texttt{notas.csv} e \texttt{INSTRUCCIONES.pdf}, aunque
pueden variar). Estos archivos, particularmente los dos primeros, son
esenciales para que funcione el método de corrección automática que
hemos desarrollado para este curso.

La carpeta con el repartido debe bajarse y descomprimirse en el disco
duro, creando la carpeta \textbf{\texttt{rep-X}} (siendo X el número de
repartido). Usted deberá abrir el RStudio y seleccionar dicha carpeta
como su directorio de trabajo con \texttt{setwd} (o en RStudio la
combinación \textbf{Ctrl + Shift + K}). En la lección 1.4 se explica en
más detalle el uso de la función \texttt{setwd} y de los directorios de
trabajo.

\subsection{Escribiendo el código}

Cada uno de estos archivos se corresponde con un ejercicio del
repartido. Se trata de archivos de texto plano (como un \texttt{.txt},
pero con una extensión diferente). En cada archivo se indica con
presición en dónde debe usted escribir su código (o modificar lo que ya
está escrito). Para que el sistema de corrección funcione bien,
\textbf{no cambie} el texto que se encuentra por fuera, incluyendo las
que se usan para indicar el inicio y el final del mismo. Nótese además
que \textbf{esto no impide} ampliar o reducir el espacio que usted tiene
para escribir, subiendo o bajando cualquiera de las ``líneas límite''
del archivo.

Cuide siempre de \textbf{no escribir} en estos archivos líneas de código
ajenas a los propósitos del ejercicio mismo. Por ejemplo, debe tener
cuidado de no dejar escritas líneas como \texttt{source(triangulo.R)}
dentro del propio archivo \texttt{triangulo.R}, ya que esto genera
problemas al momento de evaluar el ejercicio (básicamente es como
pedirle al archivo que se evalúe a sí mismo, lo que hace que R entre en
una espiral de ``autoevaluación'').

Por esta razón, recomendamos usar otro archivo (posiblemente temporal)
aparte para este tipo de comandos. En el video ``Guía para repartidos''
mostramos un ejemplo a seguir en caso de que esta explicación no le
resulte satisfactoria.

Nótese además que los cambios que se hacen al script del ejercicio son
\textbf{invisibles} para R hasta el momento en que usted \textbf{guarda}
el archivo a disco duro. Esta suele ser una fuente común de frustración
entre principiantes (y no tanto).

\subsection{Sólo sirven soluciones \emph{genéricas}}

Por ejemplo, para el segundo ejercicio del repartido 1 (para el cual se
debe completar el código del script \texttt{areaMax.R}), no es válido el
siguiente comando:

\begin{Shaded}
\begin{Highlighting}[]
\NormalTok{i <- }\DecValTok{72}
\end{Highlighting}
\end{Shaded}
Si bien para el ejemplo que se ejecuta previo al ejercicio (i.e.: las
líneas de comando que aparecen en \texttt{areaMax.R} antes del código
que usted debe editar) esta solución es la correcta, \emph{no es una
solución} \emph{\textbf{general}} para el ejercicio. Es decir, si
cambiáramos el ejemplo, entonces 72 ya no sería una solución correcta,
ya que el valor máximo dentro del vector \texttt{a} seguramente se
encuentre ubicado en otra posición del mismo. El objetivo de estos
ejercicios, es que usted desarrolle soluciones \emph{universales} para
el problema en cuestión. Encontrar soluciones universales es en donde
radica el \textbf{poder de la programación} como herramienta, y por lo
tanto es nuestro objetivo en el curso.

Por esta razón, si usted prueba usar 72 como se muestra arriba, el
sistema de corrección automática lo dará como erróneo. En general para
cualquier ejercicio, el sistema de corrección automático utiliza números
aleatorios para verificar que su solución es robusta respecto a los
casos particulares (i.e.: diferentes valores numéricos).

\subsection{Problemas de redondeo/aproximación numérica}

Ocasionalmente puede ocurrir que una solución que plantea un estudiante
tenga diferencias ínfimas con los valores esperados y resultan en
ejercicios corregidos como incompletos. Estas diferencias pueden estar
en el orden de $10 ^ {-99}$, por lo que a simple vista no se pueden ver
(R nos muestra valores redondeados al 5 decimal por defecto). El origen
de estas divergencias es la necesidad de la computadora de redondear los
números, debido a que tiene capacidades finitas y los números pueden
tener infinitos decimales.

En general tratamos de que esto no sea un problema para la corrección de
ejercicios, pero no puede descartarse que no hayamos previsto algún
caso. Si está \emph{convencido} de que esto es así, por favor
comuníquelo a través de los foros bajo la etiqueta ``fe de erratas''.

Debe notarse además que estas diferencias pueden además surgir en base a
la forma en que uno escribe las ecuaciones, sobre todo en el caso de que
se usen operaciones que ``amplifiquen'' dichas diferencias, como puede
ser la exponenciación. El siguiente es un ejemplo concreto:

\begin{Shaded}
\begin{Highlighting}[]
\NormalTok{a <- (}\DecValTok{1}\NormalTok{/}\DecValTok{7}\NormalTok{)^}\DecValTok{100}
\NormalTok{b <- (}\DecValTok{1}\NormalTok{/(}\DecValTok{7}\NormalTok{^}\DecValTok{100}\NormalTok{))}
\end{Highlighting}
\end{Shaded}
Es claro que los valores de \texttt{a} y \texttt{b} deberían ser
idénticos, ya que se obtuvieron con expresiones matemáticamente
equivalentes (i.e.: $(1 / 7) ^ {100}$ y $1 / (7 ^ {100}))$. Sin embargo
cuando buscamos verificar esto, nos sorprendemos:

\begin{Shaded}
\begin{Highlighting}[]
\NormalTok{a == b}
\end{Highlighting}
\end{Shaded}
\begin{verbatim}
## [1] FALSE
\end{verbatim}
\begin{Shaded}
\begin{Highlighting}[]
\NormalTok{a - b}
\end{Highlighting}
\end{Shaded}
\begin{verbatim}
## [1] -1.714e-99
\end{verbatim}
Aquí la evaluación de \texttt{a == b} debería dar \texttt{TRUE} y no
\texttt{FALSE}, y la última línea debería devolver un cero. Lo que
ocurre es que el orden en el que se realizan las operaciones \textbf{sí}
altera el producto en este caso, a causa de los problemas de redondeo
que antes mencionábamos. Nótese que para crear \texttt{a},
\emph{primero} se calcula $1/7$ (lo que resulta en un valor redondeado)
y \emph{luego} se eleva el resultado a la 100. Contrariamente, para el
caso de \texttt{b}, R primero eleva 7 a la 100 (lo que debería dar un
resultado exacto), y recién entonces evalúa la división (aquí es otra
vez un valor redondeado).

\subsection{Objetos nombrados}

En varios ejercicios del curso usted se encontrarán con objetos en R que
contienen nombres. Por ejemplo, las columnas de una matriz de datos
(\texttt{data.frame} es el término correcto en R) tienen nombres. Parte
de la corrección automática consiste en verificar que estos nombres
estén correctos. Esto incluye: el \textbf{orden} en el que están los
nombres así como la \textbf{capitalización} de las letras en los nombres
(i.e.: mayúsculas y minúsculas). Por esto debe uster prestar atención a
estos detalles cada vez que realice una corrección automática de los
ejercicios.

\subsection{Evaluaciones de métodos}

En algunos ejercicios (la minoría) se pide que el estudiante resuelva un
problema utilizando un método en particular (por ejemplo, usando el
operador \texttt{\%*\%} en el 3.a del repartido 1). En estos casos el
sistema de evaluación sólo dará como correcto al ejercicio si el usuario
usa efectivamente el método en cuestión (i.e.: utiliza el operador
\texttt{\%*\%} correctamente en la línea correspondiente). La idea en
general es que el estudiante tenga total libertad respecto al camino que
realiza para alcanzar una solución, pero en algunos casos excepcionales
esto no es así.

\subsection{Se evalúan varios objetos}

En muchos casos el algoritmo de evaluación no se limita a evaluar
solamente el ``objeto final'' de un script, si no que evalúa también
objetos (y/o pasos) anteriores. Por ejemplo, en el ejercicio 1.b del
repartido 1, la corrección evalúa los tres objetos, \texttt{i},
\texttt{sol} y \texttt{amax}. Por lo tanto, es importante que los
objetos se encuentren bien definidos con soluciones genéricas.

\subsection{Mecanismo de corrección automática: procedimiento}

Lo primero que debe hacer es cargar el archivo \texttt{evaluar.R} con la
función \texttt{source}, como se muestra a continuación:

\begin{Shaded}
\begin{Highlighting}[]
\KeywordTok{options}\NormalTok{(}\DataTypeTok{encoding =} \StringTok{"utf-8"}\NormalTok{)}
\KeywordTok{source}\NormalTok{(}\StringTok{"evaluar.R"}\NormalTok{)}
\end{Highlighting}
\end{Shaded}
Ejecutado correctamente, la siguiente frase debería verse en la consola:

\begin{verbatim}
Archivo de codigo fuente cargado correctamente
\end{verbatim}
En caso de que ocurra un error o se vea otro mensaje en la consola,
verifique que los archivos se descomprimieron correctamente y que usted
está trabajando en la carpeta correspondiente con el comando
\texttt{getwd()}.

Usted trabajará modificando los contenidos de dichos archivos con
RStudio (u otro programa de su preferencia) según las consignas que se
describen en cada repartido. Luego de terminar cada ejercicio y
\textbf{guardando el archivo} correspondiente en el disco duro, usted
podrá verificar rápidamente si su respuesta es correcta ejecutando el
comando:

\begin{Shaded}
\begin{Highlighting}[]
\KeywordTok{evaluar}\NormalTok{()}
\end{Highlighting}
\end{Shaded}
En todo momento podrá verificar su puntaje con la función
\texttt{verNotas()}. Una vez terminados los ejercicios del repartido,
usted deberá subir el archivo ''datos'' (sin extensión), incluido en la
carpeta ''rep-X'', a la
\href{http://eva.universidad.edu.uy/mod/assignment/view.php?id=93616}{sección
de entregas} de la portada del curso en la plataforma EVA.

\subsection{Sistema de puntaje}

Cada ejercicio en un repartido vale un punto, sin valores intermedios.
Es decir, el resultado es binario: 0 o 1 punto por ejercicio. Los
repartidos tienen ejercicios \textbf{obligatorios} pero también
ejercicios \emph{opcionales}. El puntaje total de cada repartido se
calcula como el porcentaje de puntos obtenidos respecto al total de
ejercicios \emph{obligatorios} del mismo. Por lo tanto, es posible
obtener notas mayores al 100\%. Por ejemplo: si hay 6 ejercicios
obligatorios y 2 opcionales, entonces 6 puntos equivalen a un 100\% y 8
puntos a 133\% ($nota = (puntos / 6) \cdot 100$).

Nótese que de todas formas, en la nota final del curso no se permitirán
porcentajes superiores al 100\%, en acordancia con el sistema de notas
de la UdelaR. Este sistema es simplemente una forma de motivar y poder
subir el promedio general de cada estudiante.

\subsection{Código de Honor}

Si bien animamos a que trabaje en equipos y que haya un intercambio
fluido en los foros del curso, es fundamental que las respuestas a los
cuestionarios y ejercicios de programación sean fruto del trabajo
individual. En particular, consideramos necesario que no utilize el
código creado por sus compañeros, si no que debe programar sus propias
instrucciones, ya que de lo contrario supone un sabotaje a su propio
proceso de aprendizaje. Esto implica también evitar, en la medida de lo
posible, exponer el código propio a sus colegas. Como profesores estamos
comprometidos a dar nuestro mayor esfuerzo para dar las herramientas y
explicaciones adecuadas a fin de que pueda encontrar su propio camino
para resolver los ejercicios.

En casos de planteos de dudas a través del foro, en los que considere
que es imposible expresar un problema sin exponer su própio código,
entonces es aceptable hacerlo. De todas formas en estos casos es
preferible que envíe su código por correo electrónico directamente a un
profesor, explicando la problemática.

\end{document}
